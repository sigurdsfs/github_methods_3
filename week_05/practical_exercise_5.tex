% Options for packages loaded elsewhere
\PassOptionsToPackage{unicode}{hyperref}
\PassOptionsToPackage{hyphens}{url}
%
\documentclass[
]{article}
\usepackage{amsmath,amssymb}
\usepackage{lmodern}
\usepackage{ifxetex,ifluatex}
\ifnum 0\ifxetex 1\fi\ifluatex 1\fi=0 % if pdftex
  \usepackage[T1]{fontenc}
  \usepackage[utf8]{inputenc}
  \usepackage{textcomp} % provide euro and other symbols
\else % if luatex or xetex
  \usepackage{unicode-math}
  \defaultfontfeatures{Scale=MatchLowercase}
  \defaultfontfeatures[\rmfamily]{Ligatures=TeX,Scale=1}
\fi
% Use upquote if available, for straight quotes in verbatim environments
\IfFileExists{upquote.sty}{\usepackage{upquote}}{}
\IfFileExists{microtype.sty}{% use microtype if available
  \usepackage[]{microtype}
  \UseMicrotypeSet[protrusion]{basicmath} % disable protrusion for tt fonts
}{}
\makeatletter
\@ifundefined{KOMAClassName}{% if non-KOMA class
  \IfFileExists{parskip.sty}{%
    \usepackage{parskip}
  }{% else
    \setlength{\parindent}{0pt}
    \setlength{\parskip}{6pt plus 2pt minus 1pt}}
}{% if KOMA class
  \KOMAoptions{parskip=half}}
\makeatother
\usepackage{xcolor}
\IfFileExists{xurl.sty}{\usepackage{xurl}}{} % add URL line breaks if available
\IfFileExists{bookmark.sty}{\usepackage{bookmark}}{\usepackage{hyperref}}
\hypersetup{
  pdftitle={practical\_exercise\_5, Methods 3, 2021, autumn semester},
  pdfauthor={Sigurd Fyhn Sørensen},
  hidelinks,
  pdfcreator={LaTeX via pandoc}}
\urlstyle{same} % disable monospaced font for URLs
\usepackage[margin=1in]{geometry}
\usepackage{color}
\usepackage{fancyvrb}
\newcommand{\VerbBar}{|}
\newcommand{\VERB}{\Verb[commandchars=\\\{\}]}
\DefineVerbatimEnvironment{Highlighting}{Verbatim}{commandchars=\\\{\}}
% Add ',fontsize=\small' for more characters per line
\usepackage{framed}
\definecolor{shadecolor}{RGB}{248,248,248}
\newenvironment{Shaded}{\begin{snugshade}}{\end{snugshade}}
\newcommand{\AlertTok}[1]{\textcolor[rgb]{0.94,0.16,0.16}{#1}}
\newcommand{\AnnotationTok}[1]{\textcolor[rgb]{0.56,0.35,0.01}{\textbf{\textit{#1}}}}
\newcommand{\AttributeTok}[1]{\textcolor[rgb]{0.77,0.63,0.00}{#1}}
\newcommand{\BaseNTok}[1]{\textcolor[rgb]{0.00,0.00,0.81}{#1}}
\newcommand{\BuiltInTok}[1]{#1}
\newcommand{\CharTok}[1]{\textcolor[rgb]{0.31,0.60,0.02}{#1}}
\newcommand{\CommentTok}[1]{\textcolor[rgb]{0.56,0.35,0.01}{\textit{#1}}}
\newcommand{\CommentVarTok}[1]{\textcolor[rgb]{0.56,0.35,0.01}{\textbf{\textit{#1}}}}
\newcommand{\ConstantTok}[1]{\textcolor[rgb]{0.00,0.00,0.00}{#1}}
\newcommand{\ControlFlowTok}[1]{\textcolor[rgb]{0.13,0.29,0.53}{\textbf{#1}}}
\newcommand{\DataTypeTok}[1]{\textcolor[rgb]{0.13,0.29,0.53}{#1}}
\newcommand{\DecValTok}[1]{\textcolor[rgb]{0.00,0.00,0.81}{#1}}
\newcommand{\DocumentationTok}[1]{\textcolor[rgb]{0.56,0.35,0.01}{\textbf{\textit{#1}}}}
\newcommand{\ErrorTok}[1]{\textcolor[rgb]{0.64,0.00,0.00}{\textbf{#1}}}
\newcommand{\ExtensionTok}[1]{#1}
\newcommand{\FloatTok}[1]{\textcolor[rgb]{0.00,0.00,0.81}{#1}}
\newcommand{\FunctionTok}[1]{\textcolor[rgb]{0.00,0.00,0.00}{#1}}
\newcommand{\ImportTok}[1]{#1}
\newcommand{\InformationTok}[1]{\textcolor[rgb]{0.56,0.35,0.01}{\textbf{\textit{#1}}}}
\newcommand{\KeywordTok}[1]{\textcolor[rgb]{0.13,0.29,0.53}{\textbf{#1}}}
\newcommand{\NormalTok}[1]{#1}
\newcommand{\OperatorTok}[1]{\textcolor[rgb]{0.81,0.36,0.00}{\textbf{#1}}}
\newcommand{\OtherTok}[1]{\textcolor[rgb]{0.56,0.35,0.01}{#1}}
\newcommand{\PreprocessorTok}[1]{\textcolor[rgb]{0.56,0.35,0.01}{\textit{#1}}}
\newcommand{\RegionMarkerTok}[1]{#1}
\newcommand{\SpecialCharTok}[1]{\textcolor[rgb]{0.00,0.00,0.00}{#1}}
\newcommand{\SpecialStringTok}[1]{\textcolor[rgb]{0.31,0.60,0.02}{#1}}
\newcommand{\StringTok}[1]{\textcolor[rgb]{0.31,0.60,0.02}{#1}}
\newcommand{\VariableTok}[1]{\textcolor[rgb]{0.00,0.00,0.00}{#1}}
\newcommand{\VerbatimStringTok}[1]{\textcolor[rgb]{0.31,0.60,0.02}{#1}}
\newcommand{\WarningTok}[1]{\textcolor[rgb]{0.56,0.35,0.01}{\textbf{\textit{#1}}}}
\usepackage{graphicx}
\makeatletter
\def\maxwidth{\ifdim\Gin@nat@width>\linewidth\linewidth\else\Gin@nat@width\fi}
\def\maxheight{\ifdim\Gin@nat@height>\textheight\textheight\else\Gin@nat@height\fi}
\makeatother
% Scale images if necessary, so that they will not overflow the page
% margins by default, and it is still possible to overwrite the defaults
% using explicit options in \includegraphics[width, height, ...]{}
\setkeys{Gin}{width=\maxwidth,height=\maxheight,keepaspectratio}
% Set default figure placement to htbp
\makeatletter
\def\fps@figure{htbp}
\makeatother
\setlength{\emergencystretch}{3em} % prevent overfull lines
\providecommand{\tightlist}{%
  \setlength{\itemsep}{0pt}\setlength{\parskip}{0pt}}
\setcounter{secnumdepth}{-\maxdimen} % remove section numbering
\ifluatex
  \usepackage{selnolig}  % disable illegal ligatures
\fi

\title{practical\_exercise\_5, Methods 3, 2021, autumn semester}
\author{Sigurd Fyhn Sørensen}
\date{13-10-2021}

\begin{document}
\maketitle

\hypertarget{exercises-and-objectives}{%
\section{Exercises and objectives}\label{exercises-and-objectives}}

The objectives of the exercises of this assignment are based on:
\url{https://doi.org/10.1016/j.concog.2019.03.007}

\begin{enumerate}
\def\labelenumi{\arabic{enumi})}
\setcounter{enumi}{3}
\tightlist
\item
  Download and organise the data from experiment 1\\
\item
  Use log-likelihood ratio tests to evaluate logistic regression
  models\\
\item
  Test linear hypotheses\\
\item
  Estimate psychometric functions for the Perceptual Awareness Scale and
  evaluate them
\end{enumerate}

REMEMBER: In your report, make sure to include code that can reproduce
the answers requested in the exercises below (\textbf{MAKE A KNITTED
VERSION})\\
REMEMBER: This is part 2 of Assignment 2 and will be part of your final
portfolio

\hypertarget{exercise-4---download-and-organise-the-data-from-experiment-1}{%
\section{EXERCISE 4 - Download and organise the data from experiment
1}\label{exercise-4---download-and-organise-the-data-from-experiment-1}}

\begin{verbatim}
## Installing package into 'C:/Users/sigur/OneDrive/Documents/R/win-library/4.1'
## (as 'lib' is unspecified)
\end{verbatim}

\begin{verbatim}
## Warning: package 'carret' is not available for this version of R
## 
## A version of this package for your version of R might be available elsewhere,
## see the ideas at
## https://cran.r-project.org/doc/manuals/r-patched/R-admin.html#Installing-packages
\end{verbatim}

\begin{verbatim}
## Warning: unable to access index for repository http://www.stats.ox.ac.uk/pub/RWin/bin/windows/contrib/4.1:
##   cannot open URL 'http://www.stats.ox.ac.uk/pub/RWin/bin/windows/contrib/4.1/PACKAGES'
\end{verbatim}

\begin{verbatim}
## Warning: 'BiocManager' not available.  Could not check Bioconductor.
## 
## Please use `install.packages('BiocManager')` and then retry.
\end{verbatim}

\begin{verbatim}
## Warning in p_install(package, character.only = TRUE, ...):
\end{verbatim}

\begin{verbatim}
## Warning in library(package, lib.loc = lib.loc, character.only = TRUE,
## logical.return = TRUE, : there is no package called 'carret'
\end{verbatim}

\begin{verbatim}
## Installing package into 'C:/Users/sigur/OneDrive/Documents/R/win-library/4.1'
## (as 'lib' is unspecified)
\end{verbatim}

\begin{verbatim}
## Warning: package 'TinyTex' is not available for this version of R
## 
## A version of this package for your version of R might be available elsewhere,
## see the ideas at
## https://cran.r-project.org/doc/manuals/r-patched/R-admin.html#Installing-packages
\end{verbatim}

\begin{verbatim}
## Warning: Perhaps you meant 'tinytex' ?
\end{verbatim}

\begin{verbatim}
## Warning: unable to access index for repository http://www.stats.ox.ac.uk/pub/RWin/bin/windows/contrib/4.1:
##   cannot open URL 'http://www.stats.ox.ac.uk/pub/RWin/bin/windows/contrib/4.1/PACKAGES'
\end{verbatim}

\begin{verbatim}
## Warning: 'BiocManager' not available.  Could not check Bioconductor.
## 
## Please use `install.packages('BiocManager')` and then retry.
\end{verbatim}

\begin{verbatim}
## Warning in p_install(package, character.only = TRUE, ...):
\end{verbatim}

\begin{verbatim}
## Warning in library(package, lib.loc = lib.loc, character.only = TRUE,
## logical.return = TRUE, : there is no package called 'TinyTex'
\end{verbatim}

\begin{verbatim}
## Installing package into 'C:/Users/sigur/OneDrive/Documents/R/win-library/4.1'
## (as 'lib' is unspecified)
\end{verbatim}

\begin{verbatim}
## Warning: package 'reabulk' is not available for this version of R
## 
## A version of this package for your version of R might be available elsewhere,
## see the ideas at
## https://cran.r-project.org/doc/manuals/r-patched/R-admin.html#Installing-packages
\end{verbatim}

\begin{verbatim}
## Warning: unable to access index for repository http://www.stats.ox.ac.uk/pub/RWin/bin/windows/contrib/4.1:
##   cannot open URL 'http://www.stats.ox.ac.uk/pub/RWin/bin/windows/contrib/4.1/PACKAGES'
\end{verbatim}

\begin{verbatim}
## Warning: 'BiocManager' not available.  Could not check Bioconductor.
## 
## Please use `install.packages('BiocManager')` and then retry.
\end{verbatim}

\begin{verbatim}
## Warning in p_install(package, character.only = TRUE, ...):
\end{verbatim}

\begin{verbatim}
## Warning in library(package, lib.loc = lib.loc, character.only = TRUE,
## logical.return = TRUE, : there is no package called 'reabulk'
\end{verbatim}

\begin{verbatim}
## Warning in pacman::p_load(tidyverse, lmerTest, lme4, stargazer, carret, : Failed to install/load:
## carret, TinyTex, reabulk
\end{verbatim}

Go to \url{https://osf.io/ecxsj/files/} and download the files
associated with Experiment 1 (there should be 29).\\
The data is associated with Experiment 1 of the article at the following
DOI \url{https://doi.org/10.1016/j.concog.2019.03.007}

\begin{enumerate}
\def\labelenumi{\arabic{enumi})}
\tightlist
\item
  Put the data from all subjects into a single data frame - note that
  some of the subjects do not have the \emph{seed} variable. For these
  subjects, add this variable and make in \emph{NA} for all
  observations. (The \emph{seed} variable will not be part of the
  analysis and is not an experimental variable)
\end{enumerate}

\begin{Shaded}
\begin{Highlighting}[]
\NormalTok{df\_exp1 }\OtherTok{\textless{}{-}}\NormalTok{ readbulk}\SpecialCharTok{::}\FunctionTok{read\_bulk}\NormalTok{(}\StringTok{"experiment 1"}\NormalTok{, }\AttributeTok{extension =} \StringTok{".csv"}\NormalTok{) }
\end{Highlighting}
\end{Shaded}

\begin{verbatim}
## Reading 001.csv
\end{verbatim}

\begin{verbatim}
## Reading 002.csv
\end{verbatim}

\begin{verbatim}
## Reading 003.csv
\end{verbatim}

\begin{verbatim}
## Reading 004.csv
\end{verbatim}

\begin{verbatim}
## Reading 005.csv
\end{verbatim}

\begin{verbatim}
## Reading 006.csv
\end{verbatim}

\begin{verbatim}
## Reading 007.csv
\end{verbatim}

\begin{verbatim}
## Reading 008.csv
\end{verbatim}

\begin{verbatim}
## Reading 009.csv
\end{verbatim}

\begin{verbatim}
## Reading 010.csv
\end{verbatim}

\begin{verbatim}
## Reading 011.csv
\end{verbatim}

\begin{verbatim}
## Reading 012.csv
\end{verbatim}

\begin{verbatim}
## Reading 013.csv
\end{verbatim}

\begin{verbatim}
## Reading 014.csv
\end{verbatim}

\begin{verbatim}
## Reading 015.csv
\end{verbatim}

\begin{verbatim}
## Reading 016.csv
\end{verbatim}

\begin{verbatim}
## Reading 017.csv
\end{verbatim}

\begin{verbatim}
## Reading 018.csv
\end{verbatim}

\begin{verbatim}
## Reading 019.csv
\end{verbatim}

\begin{verbatim}
## Reading 020.csv
\end{verbatim}

\begin{verbatim}
## Reading 021.csv
\end{verbatim}

\begin{verbatim}
## Reading 022.csv
\end{verbatim}

\begin{verbatim}
## Reading 023.csv
\end{verbatim}

\begin{verbatim}
## Reading 024.csv
\end{verbatim}

\begin{verbatim}
## Reading 025.csv
\end{verbatim}

\begin{verbatim}
## Reading 026.csv
\end{verbatim}

\begin{verbatim}
## Reading 027.csv
\end{verbatim}

\begin{verbatim}
## Reading 028.csv
\end{verbatim}

\begin{verbatim}
## Reading 029.csv
\end{verbatim}

\begin{Shaded}
\begin{Highlighting}[]
\FunctionTok{set.seed}\NormalTok{(}\DecValTok{3215}\NormalTok{)}
\end{Highlighting}
\end{Shaded}

\begin{verbatim}
i. Factorise the variables that need factorising  
\end{verbatim}

I identified some na's in the seed variable.

\begin{Shaded}
\begin{Highlighting}[]
\FunctionTok{sum}\NormalTok{(}\FunctionTok{is.na}\NormalTok{(df\_exp1}\SpecialCharTok{$}\NormalTok{seed))}
\end{Highlighting}
\end{Shaded}

\begin{verbatim}
## [1] 2646
\end{verbatim}

precisely 2646. So no need to change anything.

\begin{Shaded}
\begin{Highlighting}[]
\NormalTok{df\_exp1 }\OtherTok{\textless{}{-}}\NormalTok{ df\_exp1 }\SpecialCharTok{\%\textgreater{}\%} 
  \FunctionTok{mutate}\NormalTok{(}\AttributeTok{right\_answer =} \FunctionTok{if\_else}\NormalTok{(target.type }\SpecialCharTok{==} \StringTok{"odd"} \SpecialCharTok{\&}\NormalTok{ obj.resp }\SpecialCharTok{==} \StringTok{"o"} \SpecialCharTok{|}\NormalTok{ target.type }\SpecialCharTok{==} \StringTok{"even"} \SpecialCharTok{\&}\NormalTok{ obj.resp }\SpecialCharTok{==} \StringTok{"e"}\NormalTok{,}\DecValTok{1}\NormalTok{,}\DecValTok{0}\NormalTok{)) }\SpecialCharTok{\%\textgreater{}\%} 
  \FunctionTok{mutate}\NormalTok{(}\AttributeTok{right\_answer =} \FunctionTok{as.numeric}\NormalTok{(right\_answer))}


\NormalTok{df\_exp1 }\OtherTok{\textless{}{-}}\NormalTok{ df\_exp1 }\SpecialCharTok{\%\textgreater{}\%} 
  \FunctionTok{mutate}\NormalTok{(}\AttributeTok{right\_answer =} \FunctionTok{as.numeric}\NormalTok{(right\_answer)) }\SpecialCharTok{\%\textgreater{}\%} 
  \FunctionTok{mutate}\NormalTok{(}\AttributeTok{subject =} \FunctionTok{as.factor}\NormalTok{(subject)) }\SpecialCharTok{\%\textgreater{}\%} 
  \FunctionTok{mutate}\NormalTok{(}\AttributeTok{task =} \FunctionTok{as.factor}\NormalTok{(task)) }\SpecialCharTok{\%\textgreater{}\%} 
  \FunctionTok{mutate}\NormalTok{(}\AttributeTok{target.type =} \FunctionTok{as.factor}\NormalTok{ (target.type)) }\SpecialCharTok{\%\textgreater{}\%} 
  \FunctionTok{mutate}\NormalTok{(}\AttributeTok{trial.type =} \FunctionTok{as.factor}\NormalTok{(trial.type)) }\SpecialCharTok{\%\textgreater{}\%} 
  \FunctionTok{mutate}\NormalTok{(}\AttributeTok{pas =} \FunctionTok{as.factor}\NormalTok{(pas))}
\end{Highlighting}
\end{Shaded}

\begin{verbatim}
ii. Remove the practice trials from the dataset (see the _trial.type_ variable)
\end{verbatim}

\begin{Shaded}
\begin{Highlighting}[]
\FunctionTok{levels}\NormalTok{(df\_exp1}\SpecialCharTok{$}\NormalTok{trial.type)}
\end{Highlighting}
\end{Shaded}

\begin{verbatim}
## [1] "experiment" "practice"
\end{verbatim}

\begin{Shaded}
\begin{Highlighting}[]
\NormalTok{df\_exp1 }\OtherTok{\textless{}{-}}\NormalTok{ df\_exp1 }\SpecialCharTok{\%\textgreater{}\%} 
  \FunctionTok{filter}\NormalTok{(trial.type }\SpecialCharTok{==} \StringTok{"experiment"}\NormalTok{)}
\end{Highlighting}
\end{Shaded}

\begin{verbatim}
iii. Create a _correct_ variable  


iv. Describe how the _target.contrast_ and _target.frames_ variables differ compared to the data from part 1 of this assignment  
\end{verbatim}

\begin{verbatim}
##    Min. 1st Qu.  Median    Mean 3rd Qu.    Max. 
##     0.1     0.1     0.1     0.1     0.1     0.1
\end{verbatim}

In experiment 2 (part 1 of this assignment) target.frames = 3 in all
trials. In experiment 1 (part 2 of this assignment) target.frame = (1:6)
across trials.

In experiment 2 (part 1 of this assignment) target.contrast: Min. 1st
Qu. Median Mean 3rd Qu. Max. 0.01000 0.05683 0.06329 0.08771 0.09392
1.00000

In experiment 1 (part 2 of this assignment) target.contrast: Min. 1st
Qu. Median Mean 3rd Qu. Max. 0.1 0.1 0.1 0.1 0.1 0.1

\hypertarget{exercise-5---use-log-likelihood-ratio-tests-to-evaluate-logistic-regression-models}{%
\section{EXERCISE 5 - Use log-likelihood ratio tests to evaluate
logistic regression
models}\label{exercise-5---use-log-likelihood-ratio-tests-to-evaluate-logistic-regression-models}}

\begin{enumerate}
\def\labelenumi{\arabic{enumi})}
\tightlist
\item
  Do logistic regression - \emph{correct} as the dependent variable and
  \emph{target.frames} as the independent variable. (Make sure that you
  understand what \emph{target.frames} encode). Create two models - a
  pooled model and a partial-pooling model. The partial-pooling model
  should include a subject-specific intercept.
\end{enumerate}

\begin{Shaded}
\begin{Highlighting}[]
\NormalTok{m5.}\FloatTok{1.1} \OtherTok{\textless{}{-}} \FunctionTok{glmer}\NormalTok{(right\_answer }\SpecialCharTok{\textasciitilde{}}\NormalTok{ target.frames }\SpecialCharTok{+}\NormalTok{ (}\DecValTok{1}\SpecialCharTok{|}\NormalTok{subject)}\SpecialCharTok{+}\NormalTok{ (}\DecValTok{1}\SpecialCharTok{|}\NormalTok{trial), }\AttributeTok{data =}\NormalTok{ df\_exp1, }\AttributeTok{family =} \FunctionTok{binomial}\NormalTok{(}\AttributeTok{link =} \StringTok{"logit"}\NormalTok{))}

\NormalTok{m5.}\FloatTok{1.2} \OtherTok{\textless{}{-}} \FunctionTok{glm}\NormalTok{(right\_answer }\SpecialCharTok{\textasciitilde{}}\NormalTok{ target.frames, }\AttributeTok{data =}\NormalTok{ df\_exp1, }\AttributeTok{family =} \FunctionTok{binomial}\NormalTok{(}\AttributeTok{link =} \StringTok{"logit"}\NormalTok{))}
\end{Highlighting}
\end{Shaded}

\begin{verbatim}
i. the likelihood-function for logistic regression is: $L(p)={\displaystyle\prod_{i=1}^Np^{y_i}(1-p)^{(1-y_i)}}$ (Remember the probability mass function for the Bernoulli Distribution). Create a function that calculates the likelihood. 
\end{verbatim}

\begin{Shaded}
\begin{Highlighting}[]
\NormalTok{likelihood\_function }\OtherTok{\textless{}{-}} \ControlFlowTok{function}\NormalTok{(model, y)\{}
\NormalTok{  p }\OtherTok{=} \FunctionTok{fitted.values}\NormalTok{(model)}
\NormalTok{  temp\_vec }\OtherTok{=}\NormalTok{ p}\SpecialCharTok{\^{}}\NormalTok{y}\SpecialCharTok{*}\NormalTok{(}\DecValTok{1}\SpecialCharTok{{-}}\NormalTok{p)}\SpecialCharTok{\^{}}\NormalTok{(}\DecValTok{1}\SpecialCharTok{{-}}\NormalTok{y)}
  \FunctionTok{return}\NormalTok{(}\FunctionTok{prod}\NormalTok{(temp\_vec))}
\NormalTok{\}}
\end{Highlighting}
\end{Shaded}

\begin{verbatim}
ii. the log-likelihood-function for logistic regression is: $l(p) = {\displaystyle\sum_{i=1}^N}[y_i\ln{p}+(1-y_i)\ln{(1-p)}$. Create a function that calculates the log-likelihood  
\end{verbatim}

\begin{Shaded}
\begin{Highlighting}[]
\NormalTok{log\_likelihood\_function }\OtherTok{\textless{}{-}} \ControlFlowTok{function}\NormalTok{(model, y)\{}
\NormalTok{  p }\OtherTok{=} \FunctionTok{fitted.values}\NormalTok{(model)}
\NormalTok{  temp\_vec }\OtherTok{=}\NormalTok{ y }\SpecialCharTok{*} \FunctionTok{log}\NormalTok{(p) }\SpecialCharTok{+}\NormalTok{ (}\DecValTok{1}\SpecialCharTok{{-}}\NormalTok{y) }\SpecialCharTok{*} \FunctionTok{log}\NormalTok{(}\DecValTok{1}\SpecialCharTok{{-}}\NormalTok{p)}
  \FunctionTok{return}\NormalTok{(}\FunctionTok{sum}\NormalTok{(temp\_vec))}
\NormalTok{\}}
\end{Highlighting}
\end{Shaded}

\begin{verbatim}
iii. apply both functions to the pooling model you just created. Make sure that the log-likelihood matches what is returned from the _logLik_ function for the pooled model. Does the likelihood-function return a value that is surprising? Why is the log-likelihood preferable when working with computers with limited precision?  
\end{verbatim}

\begin{Shaded}
\begin{Highlighting}[]
\FunctionTok{likelihood\_function}\NormalTok{(m5.}\FloatTok{1.2}\NormalTok{, df\_exp1}\SpecialCharTok{$}\NormalTok{right\_answer)}
\end{Highlighting}
\end{Shaded}

\begin{verbatim}
## [1] 0
\end{verbatim}

\begin{Shaded}
\begin{Highlighting}[]
\FunctionTok{log\_likelihood\_function}\NormalTok{(m5.}\FloatTok{1.2}\NormalTok{, df\_exp1}\SpecialCharTok{$}\NormalTok{right\_answer)}
\end{Highlighting}
\end{Shaded}

\begin{verbatim}
## [1] -10865.25
\end{verbatim}

\begin{verbatim}
iv. now show that the log-likelihood is a little off when applied to the partial pooling model - (the likelihood function is different for the multilevel function - see section 2.1 of https://www.researchgate.net/profile/Douglas-Bates/publication/2753537_Computational_Methods_for_Multilevel_Modelling/links/00b4953b4108d73427000000/Computational-Methods-for-Multilevel-Modelling.pdf if you are interested)  
\end{verbatim}

\begin{Shaded}
\begin{Highlighting}[]
\FunctionTok{likelihood\_function}\NormalTok{(m5.}\FloatTok{1.1}\NormalTok{, df\_exp1}\SpecialCharTok{$}\NormalTok{right\_answer)}
\end{Highlighting}
\end{Shaded}

\begin{verbatim}
## [1] 0
\end{verbatim}

\begin{Shaded}
\begin{Highlighting}[]
\FunctionTok{log\_likelihood\_function}\NormalTok{(m5.}\FloatTok{1.1}\NormalTok{, df\_exp1}\SpecialCharTok{$}\NormalTok{right\_answer)}
\end{Highlighting}
\end{Shaded}

\begin{verbatim}
## [1] -10563.5
\end{verbatim}

\begin{Shaded}
\begin{Highlighting}[]
\FunctionTok{logLik}\NormalTok{(m5.}\FloatTok{1.1}\NormalTok{)}
\end{Highlighting}
\end{Shaded}

\begin{verbatim}
## 'log Lik.' -10622.03 (df=4)
\end{verbatim}

\begin{enumerate}
\def\labelenumi{\arabic{enumi})}
\setcounter{enumi}{1}
\tightlist
\item
  Use log-likelihood ratio tests to argue for the addition of predictor
  variables, start from the null model,
  \texttt{glm(correct\ \textasciitilde{}\ 1,\ \textquotesingle{}binomial\textquotesingle{},\ data)},
  then add subject-level intercepts, then add a group-level effect of
  \emph{target.frames} and finally add subject-level slopes for
  \emph{target.frames}. Also assess whether or not a correlation between
  the subject-level slopes and the subject-level intercepts should be
  included.
\end{enumerate}

\begin{Shaded}
\begin{Highlighting}[]
\NormalTok{m5.}\FloatTok{2.1} \OtherTok{\textless{}{-}} \FunctionTok{glm}\NormalTok{(right\_answer }\SpecialCharTok{\textasciitilde{}} \DecValTok{1}\NormalTok{, }\AttributeTok{family =} \FunctionTok{binomial}\NormalTok{(}\AttributeTok{link =} \StringTok{"logit"}\NormalTok{), }\AttributeTok{data =}\NormalTok{ df\_exp1)}
\NormalTok{m5.}\FloatTok{2.2} \OtherTok{\textless{}{-}} \FunctionTok{glm}\NormalTok{(right\_answer }\SpecialCharTok{\textasciitilde{}}\NormalTok{ target.frames , }\AttributeTok{family =} \FunctionTok{binomial}\NormalTok{(}\AttributeTok{link=} \StringTok{"logit"}\NormalTok{), }\AttributeTok{data =}\NormalTok{ df\_exp1)}
\NormalTok{m5.}\FloatTok{2.3} \OtherTok{\textless{}{-}} \FunctionTok{glmer}\NormalTok{(right\_answer }\SpecialCharTok{\textasciitilde{}}\NormalTok{ target.frames }\SpecialCharTok{+}\NormalTok{ (}\DecValTok{1}\SpecialCharTok{|}\NormalTok{subject), }\AttributeTok{family =} \FunctionTok{binomial}\NormalTok{(}\AttributeTok{link =} \StringTok{"logit"}\NormalTok{), }\AttributeTok{data =}\NormalTok{ df\_exp1)}

\NormalTok{m5.}\FloatTok{2.4} \OtherTok{\textless{}{-}} \FunctionTok{glmer}\NormalTok{(right\_answer }\SpecialCharTok{\textasciitilde{}}\NormalTok{ target.frames  }\SpecialCharTok{+}\NormalTok{ (}\DecValTok{1}\SpecialCharTok{+}\NormalTok{ target.frames}\SpecialCharTok{||}\NormalTok{subject), }\AttributeTok{family =} \FunctionTok{binomial}\NormalTok{(}\AttributeTok{link =} \StringTok{"logit"}\NormalTok{), }\AttributeTok{data =}\NormalTok{ df\_exp1)}
\FunctionTok{summary}\NormalTok{(m5.}\FloatTok{2.4}\NormalTok{)}
\end{Highlighting}
\end{Shaded}

\begin{verbatim}
## Generalized linear mixed model fit by maximum likelihood (Laplace
##   Approximation) [glmerMod]
##  Family: binomial  ( logit )
## Formula: right_answer ~ target.frames + (1 + target.frames || subject)
##    Data: df_exp1
## 
##      AIC      BIC   logLik deviance df.resid 
##  20928.6  20961.1 -10460.3  20920.6    25040 
## 
## Scaled residuals: 
##      Min       1Q   Median       3Q      Max 
## -14.1977   0.0898   0.2470   0.4894   1.3292 
## 
## Random effects:
##  Groups    Name          Variance Std.Dev.
##  subject   (Intercept)   0.0360   0.1897  
##  subject.1 target.frames 0.0366   0.1913  
## Number of obs: 25044, groups:  subject, 29
## 
## Fixed effects:
##               Estimate Std. Error z value Pr(>|z|)    
## (Intercept)   -1.06926    0.05082  -21.04   <2e-16 ***
## target.frames  0.82207    0.03817   21.54   <2e-16 ***
## ---
## Signif. codes:  0 '***' 0.001 '**' 0.01 '*' 0.05 '.' 0.1 ' ' 1
## 
## Correlation of Fixed Effects:
##             (Intr)
## target.frms -0.230
\end{verbatim}

Following the guideline of the lme4 creators approach to selecting
correlated or uncorrelated random effects in your model.
\url{https://cran.r-project.org/web/packages/lme4/vignettes/lmer.pdf}
(page. 7) The correlation between target.frames and intercept is fairly
high (-0.23) therefore this needs to be specified in our model by
specifying (x\textbar y). This isn't assessed by log-likelihood but
rather correlation test between slope estimates and intercept.

\begin{Shaded}
\begin{Highlighting}[]
\FunctionTok{anova}\NormalTok{( m5.}\FloatTok{2.4}\NormalTok{, m5.}\FloatTok{2.3}\NormalTok{, m5.}\FloatTok{2.1}\NormalTok{, m5.}\FloatTok{2.3}\NormalTok{)}
\end{Highlighting}
\end{Shaded}

\begin{verbatim}
## Warning in anova.merMod(m5.2.4, m5.2.3, m5.2.1, m5.2.3): failed to find model
## names, assigning generic names
\end{verbatim}

\begin{verbatim}
## Data: df_exp1
## Models:
## MODEL3: right_answer ~ 1
## MODEL2: right_answer ~ target.frames + (1 | subject)
## MODEL4: right_answer ~ target.frames + (1 | subject)
## MODEL1: right_answer ~ target.frames + (1 + target.frames || subject)
##        npar   AIC   BIC logLik deviance   Chisq Df Pr(>Chisq)    
## MODEL3    1 26685 26693 -13342    26683                          
## MODEL2    3 21250 21274 -10622    21244 5439.01  2  < 2.2e-16 ***
## MODEL4    3 21250 21274 -10622    21244    0.00  0               
## MODEL1    4 20929 20961 -10460    20921  323.49  1  < 2.2e-16 ***
## ---
## Signif. codes:  0 '***' 0.001 '**' 0.01 '*' 0.05 '.' 0.1 ' ' 1
\end{verbatim}

\begin{Shaded}
\begin{Highlighting}[]
\CommentTok{\#A model calculating the log{-}likelihoood ratio test. Since all our models are nested}
\CommentTok{\# I\textquotesingle{}ve conditioned the order of variables following log{-}likelihood. If models weren\textquotesingle{}t}
\CommentTok{\#nested I would need to order by the amount of collumns in the design matrix (X+Z).}

\NormalTok{log\_like\_ratio\_test }\OtherTok{\textless{}{-}} \ControlFlowTok{function}\NormalTok{(model1, model2)\{}
\NormalTok{  ratio }\OtherTok{=} \FunctionTok{ifelse}\NormalTok{( }\FunctionTok{logLik}\NormalTok{(model2) }\SpecialCharTok{\textless{}=} \FunctionTok{logLik}\NormalTok{(model1) , }\SpecialCharTok{{-}}\DecValTok{2}\SpecialCharTok{*}\NormalTok{(}\FunctionTok{logLik}\NormalTok{(model2) }\SpecialCharTok{{-}} \FunctionTok{logLik}\NormalTok{(model1)) , (}\SpecialCharTok{{-}}\DecValTok{2}\SpecialCharTok{*}\NormalTok{(}\FunctionTok{logLik}\NormalTok{(model1) }\SpecialCharTok{{-}} \FunctionTok{logLik}\NormalTok{(model2)))) }\CommentTok{\#log{-}like{-}ratio}
  
\NormalTok{  df\_val }\OtherTok{=} \FunctionTok{abs}\NormalTok{(}\FunctionTok{df.residual}\NormalTok{(model1) }\SpecialCharTok{{-}} \FunctionTok{df.residual}\NormalTok{(model2)) }\CommentTok{\#calculate the degree of freedom}
\NormalTok{  p\_val }\OtherTok{=} \FunctionTok{pchisq}\NormalTok{(ratio, }\AttributeTok{df =}\NormalTok{ df\_val, }\AttributeTok{lower.tail =} \ConstantTok{FALSE}\NormalTok{) }\CommentTok{\#chi{-}square test}
\NormalTok{  name }\OtherTok{=} \FunctionTok{deparse}\NormalTok{(}\FunctionTok{substitute}\NormalTok{(}\FunctionTok{c}\NormalTok{(model1, model2)))}
  \FunctionTok{return}\NormalTok{(}\FunctionTok{c}\NormalTok{(}\AttributeTok{model\_comp =}\NormalTok{ name, }\AttributeTok{log\_lik\_ratio =}\NormalTok{ ratio, }\AttributeTok{p\_value =}\NormalTok{ p\_val))}
\NormalTok{\}}


\FunctionTok{log\_like\_ratio\_test}\NormalTok{(m5.}\FloatTok{2.4}\NormalTok{, m5.}\FloatTok{2.1}\NormalTok{)}
\end{Highlighting}
\end{Shaded}

\begin{verbatim}
##          model_comp       log_lik_ratio             p_value 
## "c(m5.2.4, m5.2.1)"  "5762.50166643417"                 "0"
\end{verbatim}

\begin{Shaded}
\begin{Highlighting}[]
\FunctionTok{log\_like\_ratio\_test}\NormalTok{(m5.}\FloatTok{2.4}\NormalTok{, m5.}\FloatTok{2.2}\NormalTok{)}
\end{Highlighting}
\end{Shaded}

\begin{verbatim}
##              model_comp           log_lik_ratio                 p_value 
##     "c(m5.2.4, m5.2.2)"      "809.920544481221" "1.34272902176013e-176"
\end{verbatim}

\begin{Shaded}
\begin{Highlighting}[]
\FunctionTok{log\_like\_ratio\_test}\NormalTok{(m5.}\FloatTok{2.4}\NormalTok{, m5.}\FloatTok{2.3}\NormalTok{)}
\end{Highlighting}
\end{Shaded}

\begin{verbatim}
##             model_comp          log_lik_ratio                p_value 
##    "c(m5.2.4, m5.2.3)"      "323.48672423363" "2.52017011613141e-72"
\end{verbatim}

\begin{Shaded}
\begin{Highlighting}[]
\CommentTok{\#With package}
\NormalTok{pacman}\SpecialCharTok{::}\FunctionTok{p\_load}\NormalTok{(lmtest)}
\FunctionTok{lrtest}\NormalTok{(m5.}\FloatTok{2.3}\NormalTok{ , m5.}\FloatTok{2.4}\NormalTok{)}
\end{Highlighting}
\end{Shaded}

\begin{verbatim}
## Likelihood ratio test
## 
## Model 1: right_answer ~ target.frames + (1 | subject)
## Model 2: right_answer ~ target.frames + (1 + target.frames || subject)
##   #Df LogLik Df  Chisq Pr(>Chisq)    
## 1   3 -10622                         
## 2   4 -10460  1 323.49  < 2.2e-16 ***
## ---
## Signif. codes:  0 '***' 0.001 '**' 0.01 '*' 0.05 '.' 0.1 ' ' 1
\end{verbatim}

As shown, my functions output is similar to that of the packages
lmtest::lrtest().

\begin{verbatim}
i. write a short methods section and a results section where you indicate which model you chose and the statistics relevant for that choice. Include a plot of the estimated group-level function with `xlim=c(0, 8)` that includes the estimated subject-specific functions.
\end{verbatim}

\textbf{logLik:}We performed a log-likelihood ratio test on our nested
models with the most complex model as reference level. With a
significance level of 0.05 we will reject the null-hypothesis on all
comparisons. This supports the choice of selecting the most complex ie.
model4.

\textbf{AIC:}While log-likelihood ratio does account for model
complexity by chi-square distribution varying depending on Df.
Aikai-Information-Criteria (AIC) has shown to be a better measure for
penalizing model complexity. the single level intercept model has the
lowest AIC score and will therefore be the preferred model following
AIC.

\begin{Shaded}
\begin{Highlighting}[]
\FunctionTok{ggplot}\NormalTok{(df\_exp1, }\FunctionTok{aes}\NormalTok{(}\AttributeTok{x =}\NormalTok{ target.frames, }\AttributeTok{y =}\NormalTok{ right\_answer)) }\SpecialCharTok{+} \FunctionTok{geom\_point}\NormalTok{() }\SpecialCharTok{+}   \FunctionTok{geom\_smooth}\NormalTok{(}\AttributeTok{method =} \StringTok{"glm"}\NormalTok{, }
    \AttributeTok{method.args =} \FunctionTok{list}\NormalTok{(}\AttributeTok{family =} \StringTok{"binomial"}\NormalTok{), }
    \AttributeTok{se =} \ConstantTok{FALSE}\NormalTok{) }\SpecialCharTok{+} \FunctionTok{xlim}\NormalTok{(}\DecValTok{0}\NormalTok{,}\DecValTok{8}\NormalTok{)}
\end{Highlighting}
\end{Shaded}

\begin{verbatim}
## `geom_smooth()` using formula 'y ~ x'
\end{verbatim}

\includegraphics{practical_exercise_5_files/figure-latex/unnamed-chunk-14-1.pdf}

Above plot gives a nice generalized idea of the logistic curve. If we
were to plot the partial-pooling aspect of the model the below plot
would be more informative.

\begin{Shaded}
\begin{Highlighting}[]
\FunctionTok{ggplot}\NormalTok{(df\_exp1, }\FunctionTok{aes}\NormalTok{(}\AttributeTok{x =}\NormalTok{ target.frames, }\AttributeTok{y =} \FunctionTok{fitted}\NormalTok{(m5.}\FloatTok{2.4}\NormalTok{))) }\SpecialCharTok{+} \FunctionTok{geom\_point}\NormalTok{() }\SpecialCharTok{+} \FunctionTok{facet\_wrap}\NormalTok{(}\SpecialCharTok{\textasciitilde{}}\NormalTok{subject) }\SpecialCharTok{+} \FunctionTok{xlim}\NormalTok{(}\DecValTok{0}\NormalTok{,}\DecValTok{8}\NormalTok{) }\SpecialCharTok{+} \FunctionTok{ylim}\NormalTok{(}\DecValTok{0}\NormalTok{,}\DecValTok{1}\NormalTok{) }\SpecialCharTok{+} \FunctionTok{labs}\NormalTok{(}\AttributeTok{title =} \StringTok{"fitted value of model grouped by subject"}\NormalTok{, }\AttributeTok{y =} \StringTok{"fitted"}\NormalTok{)}
\end{Highlighting}
\end{Shaded}

\includegraphics{practical_exercise_5_files/figure-latex/unnamed-chunk-15-1.pdf}

\begin{Shaded}
\begin{Highlighting}[]
\FunctionTok{ggplot}\NormalTok{(df\_exp1, }\FunctionTok{aes}\NormalTok{(}\AttributeTok{x =}\NormalTok{ target.frames, }\AttributeTok{y =} \FunctionTok{fitted}\NormalTok{(m5.}\FloatTok{2.4}\NormalTok{))) }\SpecialCharTok{+} \FunctionTok{geom\_line}\NormalTok{(}\FunctionTok{aes}\NormalTok{(}\AttributeTok{colour =}\NormalTok{ subject)) }\SpecialCharTok{+} \FunctionTok{labs}\NormalTok{(}\AttributeTok{y =} \StringTok{"predicted propability of right answer"}\NormalTok{) }\SpecialCharTok{+} \FunctionTok{facet\_wrap}\NormalTok{(}\SpecialCharTok{\textasciitilde{}}\NormalTok{subject) }\SpecialCharTok{+} \FunctionTok{xlim}\NormalTok{(}\DecValTok{0}\NormalTok{,}\DecValTok{8}\NormalTok{)}
\end{Highlighting}
\end{Shaded}

\includegraphics{practical_exercise_5_files/figure-latex/unnamed-chunk-15-2.pdf}

While graphs are nice numbers can also be very informative in
illustrating the spread in our beta parameters b0 and b1.

\begin{Shaded}
\begin{Highlighting}[]
\CommentTok{\#get coef }
\NormalTok{coef\_m5.}\FloatTok{2.4} \OtherTok{\textless{}{-}} \FunctionTok{coef}\NormalTok{(m5.}\FloatTok{2.4}\NormalTok{)}
\NormalTok{slope }\OtherTok{\textless{}{-}} \FunctionTok{invlogit}\NormalTok{(coef\_m5.}\FloatTok{2.4}\SpecialCharTok{$}\NormalTok{subject[,}\DecValTok{2}\NormalTok{])}
\NormalTok{intercept }\OtherTok{\textless{}{-}} \FunctionTok{invlogit}\NormalTok{(coef\_m5.}\FloatTok{2.4}\SpecialCharTok{$}\NormalTok{subject[,}\DecValTok{1}\NormalTok{])}
\FunctionTok{cbind}\NormalTok{(}\AttributeTok{intercept =} \FunctionTok{summary}\NormalTok{(intercept), }\AttributeTok{slope =} \FunctionTok{summary}\NormalTok{(slope))}
\end{Highlighting}
\end{Shaded}

\begin{verbatim}
##         intercept     slope
## Min.    0.1996974 0.5766057
## 1st Qu. 0.2410499 0.6678582
## Median  0.2608130 0.7015343
## Mean    0.2568566 0.6928046
## 3rd Qu. 0.2796322 0.7208950
## Max.    0.3048688 0.7456444
\end{verbatim}

\textbf{Funny Plot} This plot concatenates all subjects individual
curve. But sadly they overlap eachother alot and make it hard to
distinguish between them. If you look closely you will be able to see 29
lines. :P

\begin{Shaded}
\begin{Highlighting}[]
\NormalTok{pacman}\SpecialCharTok{::}\FunctionTok{p\_load}\NormalTok{(}\StringTok{"sjPlot"}\NormalTok{)}

\NormalTok{(pp }\OtherTok{\textless{}{-}} \FunctionTok{plot\_model}\NormalTok{(m5.}\FloatTok{2.4}\NormalTok{ ,}\AttributeTok{type=}\StringTok{"pred"}\NormalTok{,}
       \AttributeTok{terms=}\FunctionTok{c}\NormalTok{(}\StringTok{"target.frames"}\NormalTok{,}\StringTok{"subject"}\NormalTok{),}\AttributeTok{pred.type=}\StringTok{"re"}\NormalTok{))}
\end{Highlighting}
\end{Shaded}

\begin{verbatim}
## Warning in RColorBrewer::brewer.pal(n, pal): n too large, allowed maximum for palette Set1 is 9
## Returning the palette you asked for with that many colors
\end{verbatim}

\includegraphics{practical_exercise_5_files/figure-latex/unnamed-chunk-17-1.pdf}

\begin{verbatim}
ii. also include in the results section whether the fit didn't look good for any of the subjects. If so, identify those subjects in the report, and judge (no statistical test) whether their performance (accuracy) differed from that of the other subjects. Was their performance better than chance? (Use a statistical test this time) (50 %)  
\end{verbatim}

\begin{Shaded}
\begin{Highlighting}[]
\NormalTok{pacman}\SpecialCharTok{::}\FunctionTok{p\_load}\NormalTok{(caret)}
\end{Highlighting}
\end{Shaded}

Generating an idea of accuracy in the entire data-set.

\begin{Shaded}
\begin{Highlighting}[]
\NormalTok{df\_exp1}\OtherTok{\textless{}{-}}\NormalTok{ df\_exp1 }\SpecialCharTok{\%\textgreater{}\%} 
  \FunctionTok{mutate}\NormalTok{(}\AttributeTok{fitted\_m.5.2.4 =} \FunctionTok{fitted}\NormalTok{(m5.}\FloatTok{2.4}\NormalTok{)) }\SpecialCharTok{\%\textgreater{}\%} 
  \FunctionTok{mutate}\NormalTok{(}\AttributeTok{bin\_fit\_m5.2.4 =} \FunctionTok{if\_else}\NormalTok{(fitted\_m.}\DecValTok{5}\NormalTok{.}\FloatTok{2.4} \SpecialCharTok{\textgreater{}=} \FloatTok{0.5}\NormalTok{, }\DecValTok{1}\NormalTok{, }\DecValTok{0}\NormalTok{))}

\FunctionTok{confusionMatrix}\NormalTok{(}\AttributeTok{data =} \FunctionTok{as.factor}\NormalTok{(df\_exp1}\SpecialCharTok{$}\NormalTok{bin\_fit\_m5.}\FloatTok{2.4}\NormalTok{), }\AttributeTok{reference =} \FunctionTok{as.factor}\NormalTok{(df\_exp1}\SpecialCharTok{$}\NormalTok{right\_answer))}
\end{Highlighting}
\end{Shaded}

\begin{verbatim}
## Confusion Matrix and Statistics
## 
##           Reference
## Prediction     0     1
##          0  2043  2133
##          1  3583 17285
##                                           
##                Accuracy : 0.7718          
##                  95% CI : (0.7665, 0.7769)
##     No Information Rate : 0.7754          
##     P-Value [Acc > NIR] : 0.9145          
##                                           
##                   Kappa : 0.2788          
##                                           
##  Mcnemar's Test P-Value : <2e-16          
##                                           
##             Sensitivity : 0.36314         
##             Specificity : 0.89015         
##          Pos Pred Value : 0.48922         
##          Neg Pred Value : 0.82830         
##              Prevalence : 0.22464         
##          Detection Rate : 0.08158         
##    Detection Prevalence : 0.16675         
##       Balanced Accuracy : 0.62664         
##                                           
##        'Positive' Class : 0               
## 
\end{verbatim}

We illustrate differences in subject accuracy

\begin{Shaded}
\begin{Highlighting}[]
\NormalTok{df\_exp\_temp }\OtherTok{\textless{}{-}}\NormalTok{ df\_exp1 }\SpecialCharTok{\%\textgreater{}\%} 
  \FunctionTok{group\_by}\NormalTok{(subject) }\SpecialCharTok{\%\textgreater{}\%} 
  \FunctionTok{summarise}\NormalTok{(}\AttributeTok{Percentage\_correct\_per\_subject =} \FunctionTok{sum}\NormalTok{(bin\_fit\_m5.}\FloatTok{2.4} \SpecialCharTok{==}\NormalTok{ right\_answer)}\SpecialCharTok{/}\FunctionTok{length}\NormalTok{(right\_answer)}\SpecialCharTok{*}\DecValTok{100}\NormalTok{)}

\FunctionTok{ggplot}\NormalTok{(df\_exp\_temp, }\FunctionTok{aes}\NormalTok{(}\AttributeTok{x =}\NormalTok{ subject, }\AttributeTok{y =}\NormalTok{ Percentage\_correct\_per\_subject, }\AttributeTok{fill =}\NormalTok{ subject)) }\SpecialCharTok{+} \FunctionTok{geom\_col}\NormalTok{()}
\end{Highlighting}
\end{Shaded}

\includegraphics{practical_exercise_5_files/figure-latex/unnamed-chunk-20-1.pdf}
Based on a visual inspection of predicted correct answers we can see
that subject 24 has the worst predicted accuracy score. We will
therefore perform as one-sample test to see if he performs better than
chance.

\begin{Shaded}
\begin{Highlighting}[]
\NormalTok{df\_exp1\_fil24 }\OtherTok{\textless{}{-}}\NormalTok{ df\_exp1 }\SpecialCharTok{\%\textgreater{}\%} 
  \FunctionTok{filter}\NormalTok{(subject }\SpecialCharTok{==} \DecValTok{24}\NormalTok{)}

\CommentTok{\#t{-}test on collected data}
\FunctionTok{t.test}\NormalTok{(}\FunctionTok{as.numeric}\NormalTok{(df\_exp1\_fil24}\SpecialCharTok{$}\NormalTok{right\_answer), }\AttributeTok{mu =} \FloatTok{0.5}\NormalTok{, }\AttributeTok{alternative =} \StringTok{"greater"}\NormalTok{)}
\end{Highlighting}
\end{Shaded}

\begin{verbatim}
## 
##  One Sample t-test
## 
## data:  as.numeric(df_exp1_fil24$right_answer)
## t = 4.026, df = 873, p-value = 3.083e-05
## alternative hypothesis: true mean is greater than 0.5
## 95 percent confidence interval:
##  0.5398963       Inf
## sample estimates:
## mean of x 
## 0.5675057
\end{verbatim}

\begin{Shaded}
\begin{Highlighting}[]
\CommentTok{\#t{-}test on predicted values. }
\FunctionTok{t.test}\NormalTok{(}\FunctionTok{as.numeric}\NormalTok{(df\_exp1\_fil24}\SpecialCharTok{$}\NormalTok{bin\_fit\_m5.}\FloatTok{2.4}\NormalTok{), }\AttributeTok{mu =} \FloatTok{0.5}\NormalTok{, }\AttributeTok{alternative =} \StringTok{"greater"}\NormalTok{)}
\end{Highlighting}
\end{Shaded}

\begin{verbatim}
## 
##  One Sample t-test
## 
## data:  as.numeric(df_exp1_fil24$bin_fit_m5.2.4)
## t = 10.473, df = 873, p-value < 2.2e-16
## alternative hypothesis: true mean is greater than 0.5
## 95 percent confidence interval:
##  0.6407847       Inf
## sample estimates:
## mean of x 
## 0.6670481
\end{verbatim}

The accuracy for subject 23 were above chance (50\%), (t(863) = 26.944,
p = 2.2e\^{}16)

The predicted accuracy for subject 23 were above chance (50\%), (t(863)
= 26.275, p = 2.2e\^{}16)

\begin{enumerate}
\def\labelenumi{\arabic{enumi})}
\setcounter{enumi}{2}
\item
  Now add \emph{pas} to the group-level effects - if a log-likelihood
  ratio test justifies this, also add the interaction between \emph{pas}
  and \emph{target.frames} and check whether a log-likelihood ratio test
  justifies this

  \begin{enumerate}
  \def\labelenumii{\roman{enumii}.}
  \tightlist
  \item
    if your model doesn't converge, try a different optimizer
  \end{enumerate}
\end{enumerate}

\begin{Shaded}
\begin{Highlighting}[]
\NormalTok{m5.}\FloatTok{3.1} \OtherTok{\textless{}{-}} \FunctionTok{glmer}\NormalTok{(right\_answer }\SpecialCharTok{\textasciitilde{}}\NormalTok{ pas}\SpecialCharTok{*}\NormalTok{target.frames }\SpecialCharTok{+}\NormalTok{ (target.frames}\SpecialCharTok{||}\NormalTok{subject), }\AttributeTok{data =}\NormalTok{ df\_exp1, }\AttributeTok{family =} \FunctionTok{binomial}\NormalTok{(}\AttributeTok{link =} \StringTok{"logit"}\NormalTok{), }\AttributeTok{control =} \FunctionTok{glmerControl}\NormalTok{(}\AttributeTok{optimizer=}\StringTok{"bobyqa"}\NormalTok{))}



\FunctionTok{log\_like\_ratio\_test}\NormalTok{(m5.}\FloatTok{3.1}\NormalTok{ , m5.}\FloatTok{2.4}\NormalTok{)}
\end{Highlighting}
\end{Shaded}

\begin{verbatim}
##              model_comp           log_lik_ratio                 p_value 
##     "c(m5.3.1, m5.2.4)"      "1428.81746778507" "1.39414638034747e-305"
\end{verbatim}

Our log-likelihood-ratio test support the choice of the more complex
model. (m5.3.1) including the interaction between pas:target.frames and
pas as a fixed effect.

\begin{verbatim}
ii. plot the estimated group-level functions over `xlim=c(0, 8)` for each of the four PAS-ratings - add this plot to your report (see: 5.2.i) and add a description of your chosen model. Describe how _pas_ affects accuracy together with target duration if at all. Also comment on the estimated functions' behaviour at target.frame=0 - is that behaviour reasonable?  
\end{verbatim}

\begin{Shaded}
\begin{Highlighting}[]
\NormalTok{df\_exp1 }\OtherTok{\textless{}{-}}\NormalTok{ df\_exp1 }\SpecialCharTok{\%\textgreater{}\%} 
  \FunctionTok{mutate}\NormalTok{(}\AttributeTok{pred\_interaction\_model =} \FunctionTok{fitted.values}\NormalTok{(m5.}\FloatTok{3.1}\NormalTok{))}


\NormalTok{interactions}\SpecialCharTok{::}\FunctionTok{interact\_plot}\NormalTok{(}\AttributeTok{model =}\NormalTok{ m5.}\FloatTok{3.1}\NormalTok{ , }\AttributeTok{pred =} \StringTok{"target.frames"}\NormalTok{, }\AttributeTok{modx =} \StringTok{"pas"}\NormalTok{)}
\end{Highlighting}
\end{Shaded}

\includegraphics{practical_exercise_5_files/figure-latex/unnamed-chunk-23-1.pdf}

While this plot is nice for showing the general interaction between
pas:target.frames it does not account for our random intercept and
random slop of target.frames. We will therefore continue with another
plot that can illustrate these individual differences.

\begin{Shaded}
\begin{Highlighting}[]
\FunctionTok{ggplot}\NormalTok{(df\_exp1, }\FunctionTok{aes}\NormalTok{(}\AttributeTok{x =}\NormalTok{ target.frames, }\AttributeTok{y =}\NormalTok{ pred\_interaction\_model, }\AttributeTok{col =}\NormalTok{ pas )) }\SpecialCharTok{+} \FunctionTok{geom\_point}\NormalTok{() }\SpecialCharTok{+} \FunctionTok{facet\_wrap}\NormalTok{(}\SpecialCharTok{\textasciitilde{}}\NormalTok{subject) }\SpecialCharTok{+} \FunctionTok{xlim}\NormalTok{(}\DecValTok{0}\NormalTok{,}\DecValTok{8}\NormalTok{) }\SpecialCharTok{+} \FunctionTok{ylim}\NormalTok{(}\DecValTok{0}\NormalTok{,}\DecValTok{1}\NormalTok{)}
\end{Highlighting}
\end{Shaded}

\includegraphics{practical_exercise_5_files/figure-latex/unnamed-chunk-24-1.pdf}
As shown target.frames has individual \# EXERCISE 6 - Test linear
hypotheses

In this section we are going to test different hypotheses. We assume
that we have already proved that more objective evidence (longer
duration of stimuli) is sufficient to increase accuracy in and of itself
and that more subjective evidence (higher PAS ratings) is also
sufficient to increase accuracy in and of itself.\\
We want to test a hypothesis for each of the three neighbouring
differences in PAS, i.e.~the difference between 2 and 1, the difference
between 3 and 2 and the difference between 4 and 3. More specifically,
we want to test the hypothesis that accuracy increases faster with
objective evidence if subjective evidence is higher at the same time,
i.e.~we want to test for an interaction.

\begin{enumerate}
\def\labelenumi{\arabic{enumi})}
\tightlist
\item
  Fit a model based on the following formula:
  \texttt{correct\ \textasciitilde{}\ pas\ *\ target.frames\ +\ (target.frames\ \textbar{}\ subject))}

  \begin{enumerate}
  \def\labelenumii{\roman{enumii}.}
  \tightlist
  \item
    First, use \texttt{summary} (yes, you are allowed to!) to argue that
    accuracy increases faster with objective evidence for PAS 2 than for
    PAS 1.
  \end{enumerate}
\end{enumerate}

\begin{Shaded}
\begin{Highlighting}[]
\NormalTok{coef\_OG\_scale }\OtherTok{\textless{}{-}}  \FunctionTok{data.frame}\NormalTok{(}\AttributeTok{intercept =} \FunctionTok{rep}\NormalTok{(}\ConstantTok{NA}\NormalTok{,}\DecValTok{29}\NormalTok{), }\AttributeTok{pas2 =} \FunctionTok{rep}\NormalTok{(}\ConstantTok{NA}\NormalTok{,}\DecValTok{29}\NormalTok{), }\AttributeTok{pas3 =} \FunctionTok{rep}\NormalTok{(}\ConstantTok{NA}\NormalTok{,}\DecValTok{29}\NormalTok{), }\AttributeTok{pas4 =} \FunctionTok{rep}\NormalTok{(}\ConstantTok{NA}\NormalTok{,}\DecValTok{29}\NormalTok{), }\AttributeTok{target.frames =} \FunctionTok{rep}\NormalTok{(}\ConstantTok{NA}\NormalTok{,}\DecValTok{29}\NormalTok{), }\AttributeTok{pas2\_target.frames =} \FunctionTok{rep}\NormalTok{(}\ConstantTok{NA}\NormalTok{,}\DecValTok{29}\NormalTok{), }\AttributeTok{pas3\_target.frames =} \FunctionTok{rep}\NormalTok{(}\ConstantTok{NA}\NormalTok{,}\DecValTok{29}\NormalTok{), }\AttributeTok{pas4\_target.frames =} \FunctionTok{rep}\NormalTok{(}\ConstantTok{NA}\NormalTok{,}\DecValTok{29}\NormalTok{))}

\ControlFlowTok{for}\NormalTok{ (i }\ControlFlowTok{in} \DecValTok{1}\SpecialCharTok{:}\FunctionTok{length}\NormalTok{(}\FunctionTok{coef}\NormalTok{(m5.}\FloatTok{3.1}\NormalTok{)}\SpecialCharTok{$}\NormalTok{subject))\{}
\NormalTok{  coef\_OG\_scale[,i] }\OtherTok{=} \FunctionTok{invlogit}\NormalTok{(}\FunctionTok{coef}\NormalTok{(m5.}\FloatTok{3.1}\NormalTok{)}\SpecialCharTok{$}\NormalTok{subject[,i])}
\NormalTok{\}}
\end{Highlighting}
\end{Shaded}

All beta values were transformed from log-odds to probabilities through
an inverse logit transformation. All beta values will be reported on the
probability scale for easier interpretation. '

\textbf{Intercepts} Pas is coded as a factor and we will therefore gain
an intercept for all 4 levels of pas. \emph{general baseline intercept}
\$\beta\_0 (intercept/pas1) = 0.4714, z = -2.064, p = 0.0390 \$
\emph{Individual baseline intercept per subject}
\(\beta_0 (intercept/pas1): Median = 0.47, Mean = 0.4716, sd = 0.011\)
Every subject has their own baseline/intercept probabillity of giving
the right answer when target.frames = 0 and pas = pas1.

\(\beta_1 (intercept/pas2) = 0.3604, z = -6.471, p = 9.77e-11\) added on
top of of their individual baseline for pas1
\(\beta_2 (intercept/pas3) = 0.3771, z = -3.653, p = 0.000259 ***\)
added on top of of their individual baseline for pas1
\(\beta_3 (intercept/pas4) = 0.5658, z = 1.065, p = 0.286944\) added on
top of of their individual baseline for pas1

\textbf{Slopes} We've modeled an interaction between a categorical
variable(pas1-4) and a numeric (target.frames) with a second level
effect of random slope We will therefore have a individual slope of
target.frame for each subject. \textbf{main effect}
\(\beta_4 (target.frames) = Median = 0.5282, Mean = 0.5269, sd = 0.0234\)
\textbf{interaction effect} \$\beta\_5 (target.frames:pas2) = 0.6102, z
= 13.006, p \textless{} 2e-16 \$ \$\beta\_6 (target.frames:pas3) =
0.6764, z = 16.336, p \textless{} 2e-16 \$ \$\beta\_7
(target.frames:pas4) = 0.6776, z = 10.990, p \textless{} 2e-16 \$

Currently both genereal and individual intercept are below chance
(50\%). An explanation can be found in the interpritation. Currently our
intercept represent target.frames = 0 and pas = pas1. Having been
presented with 0 frames does not make much sense in terms of the
experiment. But we would expect a 50/50 chance to guess right if you had
no information to guess by. The best solution would be to standardize
the data around target.type. Giving us a more intuitive interpritation.

Our model supports our assumption of an increase in subjective
(pas-score) and objective (target.frames) evidence significantly
increases the probability of the right answer.

Furthermore, as shown by our interaction effects increasing subjective
evidence (PAS1 -\textgreater{} PAS2/3/4) will significantly modulate
objective evidence to increase the accuracy score even faster compared
to PAS1.

\begin{enumerate}
\def\labelenumi{\arabic{enumi})}
\setcounter{enumi}{1}
\tightlist
\item
  \texttt{summary} won't allow you to test whether accuracy increases
  faster with objective evidence for PAS 3 than for PAS 2 (unless you
  use \texttt{relevel}, which you are not allowed to in this exercise).
  Instead, we'll be using the function \texttt{glht} from the
  \texttt{multcomp} package

  \begin{enumerate}
  \def\labelenumii{\roman{enumii}.}
  \tightlist
  \item
    To redo the test in 6.1.i, you can create a \emph{contrast} vector.
    This vector will have the length of the number of estimated
    group-level effects and any specific contrast you can think of can
    be specified using this. For redoing the test from 6.1.i, the code
    snippet below will do
  \end{enumerate}
\end{enumerate}

\begin{Shaded}
\begin{Highlighting}[]
\NormalTok{design }\OtherTok{\textless{}{-}} \FunctionTok{model.matrix}\NormalTok{(m5.}\FloatTok{3.1}\NormalTok{)}
\FunctionTok{head}\NormalTok{(design)}
\end{Highlighting}
\end{Shaded}

\begin{verbatim}
##   (Intercept) pas2 pas3 pas4 target.frames pas2:target.frames
## 1           1    0    0    0             3                  0
## 2           1    0    0    0             2                  0
## 3           1    0    0    0             1                  0
## 4           1    0    1    0             5                  0
## 5           1    0    1    0             6                  0
## 6           1    0    0    1             4                  0
##   pas3:target.frames pas4:target.frames
## 1                  0                  0
## 2                  0                  0
## 3                  0                  0
## 4                  5                  0
## 5                  6                  0
## 6                  0                  4
\end{verbatim}

\begin{Shaded}
\begin{Highlighting}[]
\NormalTok{pacman}\SpecialCharTok{::}\FunctionTok{p\_load}\NormalTok{(multcomp)}
\NormalTok{contrast.vectors }\OtherTok{\textless{}{-}} \FunctionTok{matrix}\NormalTok{(}\FunctionTok{c}\NormalTok{(}\DecValTok{0}\NormalTok{, }\DecValTok{0}\NormalTok{, }\DecValTok{0}\NormalTok{, }\DecValTok{0}\NormalTok{, }\SpecialCharTok{{-}}\DecValTok{1}\NormalTok{, }\DecValTok{1}\NormalTok{, }\DecValTok{0}\NormalTok{, }\DecValTok{0}\NormalTok{), }\AttributeTok{nrow=}\DecValTok{1}\NormalTok{)}
\NormalTok{ghs }\OtherTok{\textless{}{-}} \FunctionTok{glht}\NormalTok{(m5.}\FloatTok{3.1}\NormalTok{, contrast.vectors)}
\FunctionTok{print}\NormalTok{(}\FunctionTok{summary}\NormalTok{(ghs))}
\end{Highlighting}
\end{Shaded}

\begin{verbatim}
## 
##   Simultaneous Tests for General Linear Hypotheses
## 
## Fit: glmer(formula = right_answer ~ pas * target.frames + (target.frames || 
##     subject), data = df_exp1, family = binomial(link = "logit"), 
##     control = glmerControl(optimizer = "bobyqa"))
## 
## Linear Hypotheses:
##        Estimate Std. Error z value Pr(>|z|)    
## 1 == 0   0.3388     0.0573   5.912 3.38e-09 ***
## ---
## Signif. codes:  0 '***' 0.001 '**' 0.01 '*' 0.05 '.' 0.1 ' ' 1
## (Adjusted p values reported -- single-step method)
\end{verbatim}

\begin{verbatim}
ii. Now test the hypothesis that accuracy increases faster with objective evidence for PAS 3 than for PAS 2.
\end{verbatim}

\begin{Shaded}
\begin{Highlighting}[]
\CommentTok{\#Contrast matrix}


\NormalTok{contrast\_matrix\_compar }\OtherTok{\textless{}{-}} \FunctionTok{glht}\NormalTok{(m5.}\FloatTok{3.1}\NormalTok{, }\AttributeTok{linfct =} \FunctionTok{c}\NormalTok{( }\StringTok{"pas2:target.frames = {-}1"}\NormalTok{, }\StringTok{"pas3:target.frames = 1"}\NormalTok{))}
\FunctionTok{summary}\NormalTok{(contrast\_matrix\_compar) }\CommentTok{\#not a 100\% how this works. }
\end{Highlighting}
\end{Shaded}

\begin{verbatim}
## 
##   Simultaneous Tests for General Linear Hypotheses
## 
## Fit: glmer(formula = right_answer ~ pas * target.frames + (target.frames || 
##     subject), data = df_exp1, family = binomial(link = "logit"), 
##     control = glmerControl(optimizer = "bobyqa"))
## 
## Linear Hypotheses:
##                          Estimate Std. Error z value Pr(>|z|)    
## pas2:target.frames == -1  0.44821    0.03446  42.024  < 1e-10 ***
## pas3:target.frames == 1   0.73768    0.04516  -5.809 1.26e-08 ***
## ---
## Signif. codes:  0 '***' 0.001 '**' 0.01 '*' 0.05 '.' 0.1 ' ' 1
## (Adjusted p values reported -- single-step method)
\end{verbatim}

\begin{Shaded}
\begin{Highlighting}[]
\CommentTok{\#Difference between pas2:target.frames pas3:target.frames}
\NormalTok{contrast.vector }\OtherTok{\textless{}{-}} \FunctionTok{matrix}\NormalTok{(}\FunctionTok{c}\NormalTok{(}\DecValTok{0}\NormalTok{, }\DecValTok{0}\NormalTok{, }\DecValTok{0}\NormalTok{, }\DecValTok{0}\NormalTok{, }\DecValTok{0}\NormalTok{, }\SpecialCharTok{{-}}\DecValTok{1}\NormalTok{, }\DecValTok{1}\NormalTok{, }\DecValTok{0}\NormalTok{), }\AttributeTok{nrow=}\DecValTok{1}\NormalTok{)}
\NormalTok{gh }\OtherTok{\textless{}{-}} \FunctionTok{glht}\NormalTok{(m5.}\FloatTok{3.1}\NormalTok{, contrast.vector)}
\FunctionTok{print}\NormalTok{(}\FunctionTok{summary}\NormalTok{(gh))}
\end{Highlighting}
\end{Shaded}

\begin{verbatim}
## 
##   Simultaneous Tests for General Linear Hypotheses
## 
## Fit: glmer(formula = right_answer ~ pas * target.frames + (target.frames || 
##     subject), data = df_exp1, family = binomial(link = "logit"), 
##     control = glmerControl(optimizer = "bobyqa"))
## 
## Linear Hypotheses:
##        Estimate Std. Error z value Pr(>|z|)    
## 1 == 0  0.28948    0.04582   6.318 2.65e-10 ***
## ---
## Signif. codes:  0 '***' 0.001 '**' 0.01 '*' 0.05 '.' 0.1 ' ' 1
## (Adjusted p values reported -- single-step method)
\end{verbatim}

objective evidence is also modulated more by PAS3 than PAS2 resulting in
a faster growing accuracy by target.frames in PAS3 and than PAS2.

\begin{verbatim}
iii. Also test the hypothesis that accuracy increases faster with objective evidence for PAS 4 than for PAS 3
\end{verbatim}

\begin{Shaded}
\begin{Highlighting}[]
\CommentTok{\#Difference between pas2:target.frames pas3:target.frames}
\NormalTok{contrast.vector2 }\OtherTok{\textless{}{-}} \FunctionTok{matrix}\NormalTok{(}\FunctionTok{c}\NormalTok{(}\DecValTok{0}\NormalTok{, }\DecValTok{0}\NormalTok{, }\DecValTok{0}\NormalTok{, }\DecValTok{0}\NormalTok{, }\DecValTok{0}\NormalTok{, }\DecValTok{0}\NormalTok{, }\SpecialCharTok{{-}}\DecValTok{1}\NormalTok{, }\DecValTok{1}\NormalTok{), }\AttributeTok{nrow=}\DecValTok{1}\NormalTok{)}
\NormalTok{gh2 }\OtherTok{\textless{}{-}} \FunctionTok{glht}\NormalTok{(m5.}\FloatTok{3.1}\NormalTok{, contrast.vector2)}
\FunctionTok{print}\NormalTok{(}\FunctionTok{summary}\NormalTok{(gh2))}
\end{Highlighting}
\end{Shaded}

\begin{verbatim}
## 
##   Simultaneous Tests for General Linear Hypotheses
## 
## Fit: glmer(formula = right_answer ~ pas * target.frames + (target.frames || 
##     subject), data = df_exp1, family = binomial(link = "logit"), 
##     control = glmerControl(optimizer = "bobyqa"))
## 
## Linear Hypotheses:
##        Estimate Std. Error z value Pr(>|z|)
## 1 == 0 0.005258   0.073618   0.071    0.943
## (Adjusted p values reported -- single-step method)
\end{verbatim}

Moving from PAS3-PAS4 does not increase the effect of target.frames
significantly.

\begin{enumerate}
\def\labelenumi{\arabic{enumi})}
\setcounter{enumi}{2}
\tightlist
\item
  Finally, test that whether the difference between PAS 2 and 1 (tested
  in 6.1.i) is greater than the difference between PAS 4 and 3 (tested
  in 6.2.iii)
\end{enumerate}

\begin{Shaded}
\begin{Highlighting}[]
\CommentTok{\# beta = 0.44821   std.error =  0.03446 pas1{-}pas2}
\CommentTok{\# beta = 0.005258   std.error = 0.073618 pas3{-}pas4}
\NormalTok{confint\_gh}\OtherTok{\textless{}{-}} \FunctionTok{confint}\NormalTok{(gh)}
\NormalTok{confint\_gh2 }\OtherTok{\textless{}{-}} \FunctionTok{confint}\NormalTok{(gh2)}

\NormalTok{df\_confint }\OtherTok{\textless{}{-}} \FunctionTok{data.frame}\NormalTok{(}\AttributeTok{beta =} \FunctionTok{c}\NormalTok{(confint\_gh}\SpecialCharTok{$}\NormalTok{confint[}\DecValTok{1}\NormalTok{], confint\_gh2}\SpecialCharTok{$}\NormalTok{confint[}\DecValTok{1}\NormalTok{]), }
           \AttributeTok{upper =} \FunctionTok{c}\NormalTok{(confint\_gh}\SpecialCharTok{$}\NormalTok{confint[}\DecValTok{2}\NormalTok{], confint\_gh2}\SpecialCharTok{$}\NormalTok{confint[}\DecValTok{2}\NormalTok{]), }
           \AttributeTok{lower =} \FunctionTok{c}\NormalTok{(confint\_gh}\SpecialCharTok{$}\NormalTok{confint[}\DecValTok{3}\NormalTok{], confint\_gh2}\SpecialCharTok{$}\NormalTok{confint[}\DecValTok{3}\NormalTok{]),}
           \AttributeTok{PAS1\_2 =} \FunctionTok{c}\NormalTok{(}\DecValTok{1}\NormalTok{,}\DecValTok{0}\NormalTok{),}
           \AttributeTok{PAS3\_4 =} \FunctionTok{c}\NormalTok{(}\DecValTok{0}\NormalTok{,}\DecValTok{1}\NormalTok{))}

\FunctionTok{ggplot}\NormalTok{(df\_confint, }\FunctionTok{aes}\NormalTok{(}\AttributeTok{x =}\NormalTok{ PAS1\_2, }\AttributeTok{y =}\NormalTok{ beta)) }\SpecialCharTok{+} \FunctionTok{geom\_point}\NormalTok{() }\SpecialCharTok{+} \FunctionTok{geom\_errorbar}\NormalTok{(}\FunctionTok{aes}\NormalTok{(}\AttributeTok{x =}\NormalTok{ PAS1\_2, }\AttributeTok{ymin =}\NormalTok{ lower, }\AttributeTok{ymax =}\NormalTok{ upper), }\AttributeTok{width =} \FloatTok{0.2}\NormalTok{, }\AttributeTok{color =} \StringTok{"red"}\NormalTok{, }\AttributeTok{size =} \DecValTok{1}\NormalTok{)}
\end{Highlighting}
\end{Shaded}

\includegraphics{practical_exercise_5_files/figure-latex/unnamed-chunk-32-1.pdf}
I can not do a t-test as calculating SE\_pooled requires the SD.
\(SE_p = SD_p * sqrt(1/n_1 + 1/n_2)\). We can calculate the 95\%
confidence intervals based on SE x 1.96 for lower and high boundaries.
Our Confidence intervals does not overlap which would indicate that they
are significantly different. THe difference of \textbf{PAS1-2} is larger
than \textbf{PAS3-4}.

\hypertarget{exercise-7---estimate-psychometric-functions-for-the-perceptual-awareness-scale-and-evaluate-them}{%
\section{EXERCISE 7 - Estimate psychometric functions for the Perceptual
Awareness Scale and evaluate
them}\label{exercise-7---estimate-psychometric-functions-for-the-perceptual-awareness-scale-and-evaluate-them}}

We saw in 5.3 that the estimated functions went below chance at a target
duration of 0 frames (0 ms). This does not seem reasonable, so we will
be trying a different approach for fitting here.\\
We will fit the following function that results in a sigmoid,
\(f(x) = a + \frac {b - a} {1 + e^{\frac {c-x} {d}}}\)\\
It has four parameters: \emph{a}, which can be interpreted as the
minimum accuracy level, \emph{b}, which can be interpreted as the
maximum accuracy level, \emph{c}, which can be interpreted as the
so-called inflexion point, i.e.~where the derivative of the sigmoid
reaches its maximum and \emph{d}, which can be interpreted as the
steepness at the inflexion point. (When \emph{d} goes towards infinity,
the slope goes towards a straight line, and when it goes towards 0, the
slope goes towards a step function).

We can define a function of a residual sum of squares as below

\begin{Shaded}
\begin{Highlighting}[]
\NormalTok{RSS }\OtherTok{\textless{}{-}} \ControlFlowTok{function}\NormalTok{(data, par)}
\NormalTok{\{}
    \DocumentationTok{\#\# "dataset" should be a data.frame containing the variables x (target.frames)}
    \DocumentationTok{\#\# and y (correct)}
    
    \DocumentationTok{\#\# "par" are our four parameters (a numeric vector) }
    \DocumentationTok{\#\# par[1]=a, par[2]=b, par[3]=c, par[4]=d}
\NormalTok{    a }\OtherTok{=}\NormalTok{ par[}\DecValTok{1}\NormalTok{]}
\NormalTok{    b }\OtherTok{=}\NormalTok{ par[}\DecValTok{2}\NormalTok{]}
\NormalTok{    c }\OtherTok{=}\NormalTok{ par[}\DecValTok{3}\NormalTok{]}
\NormalTok{    d }\OtherTok{=}\NormalTok{ par[}\DecValTok{4}\NormalTok{]}
\NormalTok{    x }\OtherTok{\textless{}{-}}\NormalTok{ data}\SpecialCharTok{$}\NormalTok{x}
\NormalTok{    y }\OtherTok{\textless{}{-}}\NormalTok{ data}\SpecialCharTok{$}\NormalTok{y}
\NormalTok{    y.hat }\OtherTok{\textless{}{-}}\NormalTok{ a }\SpecialCharTok{+}\NormalTok{ ((b}\SpecialCharTok{{-}}\NormalTok{a)}\SpecialCharTok{/}\NormalTok{(}\DecValTok{1}\SpecialCharTok{+}\NormalTok{(}\FunctionTok{exp}\NormalTok{((c}\SpecialCharTok{{-}}\NormalTok{x)}\SpecialCharTok{/}\NormalTok{d))))}
\NormalTok{    RSS }\OtherTok{\textless{}{-}} \FunctionTok{sum}\NormalTok{((y }\SpecialCharTok{{-}}\NormalTok{ y.hat)}\SpecialCharTok{\^{}}\DecValTok{2}\NormalTok{)}
    \FunctionTok{return}\NormalTok{(RSS)}
\NormalTok{\}}
\end{Highlighting}
\end{Shaded}

\begin{enumerate}
\def\labelenumi{\arabic{enumi})}
\tightlist
\item
  Now, we will fit the sigmoid for the four PAS ratings for Subject 7

  \begin{enumerate}
  \def\labelenumii{\roman{enumii}.}
  \tightlist
  \item
    use the function \texttt{optim}. It returns a list that among other
    things contains the four estimated parameters. You should set the
    following arguments:\\
    \texttt{par}: you can set \emph{c} and \emph{d} as 1. Find good
    choices for \emph{a} and \emph{b} yourself (and argue why they are
    appropriate)\\
    \texttt{fn}: which function to minimise?\\
    \texttt{data}: the data frame with \emph{x}, \emph{target.frames},
    and \emph{y}, \emph{correct} in it\\
    \texttt{method}: `L-BFGS-B'\\
    \texttt{lower}: lower bounds for the four parameters, (the lowest
    value they can take), you can set \emph{c} and \emph{d} as
    \texttt{-Inf}. Find good choices for \emph{a} and \emph{b} yourself
    (and argue why they are appropriate)\\
    \texttt{upper}: upper bounds for the four parameters, (the highest
    value they can take) can set \emph{c} and \emph{d} as \texttt{Inf}.
    Find good choices for \emph{a} and \emph{b} yourself (and argue why
    they are appropriate)
  \end{enumerate}
\end{enumerate}

\begin{Shaded}
\begin{Highlighting}[]
\CommentTok{\#Optim function with target.frames}
\NormalTok{data }\OtherTok{\textless{}{-}}\NormalTok{ df\_exp1 }\SpecialCharTok{\%\textgreater{}\%}
\NormalTok{  dplyr}\SpecialCharTok{::}\FunctionTok{select}\NormalTok{(target.frames, right\_answer) }\SpecialCharTok{\%\textgreater{}\%} 
  \FunctionTok{rename}\NormalTok{(}\AttributeTok{x =}\NormalTok{ target.frames, }\AttributeTok{y =}\NormalTok{ right\_answer)}


\FunctionTok{optim}\NormalTok{(}\AttributeTok{par =} \FunctionTok{c}\NormalTok{(}\FloatTok{0.5}\NormalTok{,}\DecValTok{1}\NormalTok{,}\DecValTok{1}\NormalTok{,}\DecValTok{1}\NormalTok{), }\AttributeTok{fn =}\NormalTok{ RSS, }\AttributeTok{data =}\NormalTok{ data, }\AttributeTok{method =} \StringTok{"L{-}BFGS{-}B"}\NormalTok{, }\AttributeTok{lower =} \FunctionTok{c}\NormalTok{(}\FloatTok{0.5}\NormalTok{,}\FloatTok{0.5}\NormalTok{,}\SpecialCharTok{{-}}\ConstantTok{Inf}\NormalTok{,}\SpecialCharTok{{-}}\ConstantTok{Inf}\NormalTok{),}
      \AttributeTok{upper =} \FunctionTok{c}\NormalTok{(}\DecValTok{1}\NormalTok{,}\DecValTok{1}\NormalTok{,}\ConstantTok{Inf}\NormalTok{,}\ConstantTok{Inf}\NormalTok{))}
\end{Highlighting}
\end{Shaded}

\begin{verbatim}
## $par
## [1] 0.5077512 0.9586143 2.9380622 0.4303382
## 
## $value
## [1] 3500.344
## 
## $counts
## function gradient 
##       24       24 
## 
## $convergence
## [1] 0
## 
## $message
## [1] "CONVERGENCE: REL_REDUCTION_OF_F <= FACTR*EPSMCH"
\end{verbatim}

\begin{Shaded}
\begin{Highlighting}[]
\CommentTok{\#Optim function with pas rating. }
\NormalTok{data1 }\OtherTok{\textless{}{-}}\NormalTok{ df\_exp1 }\SpecialCharTok{\%\textgreater{}\%}
  \FunctionTok{filter}\NormalTok{(subject }\SpecialCharTok{==} \DecValTok{7}\NormalTok{) }\SpecialCharTok{\%\textgreater{}\%} \CommentTok{\#only use subject 7}
\NormalTok{  dplyr}\SpecialCharTok{::}\FunctionTok{select}\NormalTok{(pas, right\_answer) }\SpecialCharTok{\%\textgreater{}\%} 
  \FunctionTok{mutate}\NormalTok{(}\AttributeTok{pas =} \FunctionTok{as.numeric}\NormalTok{(pas)) }\SpecialCharTok{\%\textgreater{}\%} \CommentTok{\#optim function requires it to be numeric. }
  \FunctionTok{rename}\NormalTok{(}\AttributeTok{x =}\NormalTok{ pas, }\AttributeTok{y =}\NormalTok{ right\_answer)}


\NormalTok{optim\_pas }\OtherTok{\textless{}{-}} \FunctionTok{optim}\NormalTok{(}\AttributeTok{par =} \FunctionTok{c}\NormalTok{(}\FloatTok{0.5}\NormalTok{,}\DecValTok{1}\NormalTok{,}\DecValTok{1}\NormalTok{,}\DecValTok{1}\NormalTok{), }\AttributeTok{fn =}\NormalTok{ RSS, }\AttributeTok{data =}\NormalTok{ data1, }\AttributeTok{method =} \StringTok{"L{-}BFGS{-}B"}\NormalTok{, }\AttributeTok{lower =} \FunctionTok{c}\NormalTok{(}\FloatTok{0.5}\NormalTok{,}\FloatTok{0.5}\NormalTok{,}\SpecialCharTok{{-}}\ConstantTok{Inf}\NormalTok{,}\SpecialCharTok{{-}}\ConstantTok{Inf}\NormalTok{),}
      \AttributeTok{upper =} \FunctionTok{c}\NormalTok{(}\DecValTok{1}\NormalTok{,}\DecValTok{1}\NormalTok{,}\ConstantTok{Inf}\NormalTok{,}\ConstantTok{Inf}\NormalTok{))}
\NormalTok{optim\_pas}\SpecialCharTok{$}\NormalTok{par}
\end{Highlighting}
\end{Shaded}

\begin{verbatim}
## [1] 0.5000000 0.9945230 2.6256293 0.3288213
\end{verbatim}

par = c(0.5, 1,1,1,) \emph{a} will start at 0.5 with a min = 0.5 and max
= 1. \emph{b} will start at 1 with min = 0.5 and max = 1. The choice of
min and max is based on the assignments wish to avoid values below
chance. So max and min values of y can span from 0.5:1. It is imaginable
that such values could be possible so therefore they have been chosen as
the lower and upper boundraies.

Starting point is based on guesses of what the best scenario could be
and worst scenario. ie. \emph{a} = 0.5 and \emph{b} = 1.

Use the estimated parameters to compute y\_hat and plot the function.

\begin{Shaded}
\begin{Highlighting}[]
\NormalTok{sigmoid }\OtherTok{\textless{}{-}} \ControlFlowTok{function}\NormalTok{(a,b, c,d,x)\{ }
\NormalTok{  y }\OtherTok{=}\NormalTok{ a }\SpecialCharTok{+}\NormalTok{ ((b}\SpecialCharTok{{-}}\NormalTok{a)}\SpecialCharTok{/}\NormalTok{(}\DecValTok{1}\SpecialCharTok{+}\NormalTok{(}\FunctionTok{exp}\NormalTok{((c}\SpecialCharTok{{-}}\NormalTok{x)}\SpecialCharTok{/}\NormalTok{d))))}
  \FunctionTok{return}\NormalTok{(y)}
\NormalTok{\}}

\NormalTok{x\_pas\_optim }\OtherTok{\textless{}{-}} \FunctionTok{seq}\NormalTok{(}\DecValTok{0}\NormalTok{, }\DecValTok{4}\NormalTok{, }\FloatTok{0.01}\NormalTok{)}
\NormalTok{y\_pas\_optim }\OtherTok{\textless{}{-}} \FunctionTok{sigmoid}\NormalTok{(optim\_pas}\SpecialCharTok{$}\NormalTok{par[}\DecValTok{1}\NormalTok{],optim\_pas}\SpecialCharTok{$}\NormalTok{par[}\DecValTok{2}\NormalTok{],}
\NormalTok{                       optim\_pas}\SpecialCharTok{$}\NormalTok{par[}\DecValTok{3}\NormalTok{],optim\_pas}\SpecialCharTok{$}\NormalTok{par[}\DecValTok{4}\NormalTok{],x\_pas\_optim) }
\end{Highlighting}
\end{Shaded}

\begin{Shaded}
\begin{Highlighting}[]
\FunctionTok{plot}\NormalTok{(data1}\SpecialCharTok{$}\NormalTok{x, data1}\SpecialCharTok{$}\NormalTok{y, }\AttributeTok{pch =} \DecValTok{16}\NormalTok{, }\AttributeTok{xlab =} \StringTok{"Pas Rating"}\NormalTok{, }\AttributeTok{ylab =} \StringTok{"\% right answer = 1"}\NormalTok{)}
\FunctionTok{lines}\NormalTok{(x\_pas\_optim, y\_pas\_optim) }\SpecialCharTok{+} \FunctionTok{title}\NormalTok{(}\AttributeTok{main =} \StringTok{"Sigmoid function for pas{-}ratings."}\NormalTok{, }\AttributeTok{sub =} \StringTok{"plot1"}\NormalTok{)}
\end{Highlighting}
\end{Shaded}

\includegraphics{practical_exercise_5_files/figure-latex/unnamed-chunk-37-1.pdf}

\begin{verbatim}
## integer(0)
\end{verbatim}

\begin{verbatim}
ii. Plot the fits for the PAS ratings on a single plot (for subject 7) `xlim=c(0, 8)`
\end{verbatim}

\begin{Shaded}
\begin{Highlighting}[]
\CommentTok{\#Simple model to do illustrate pas ratings for subject 7}
\NormalTok{m7.}\FloatTok{1.1} \OtherTok{\textless{}{-}} \FunctionTok{glm}\NormalTok{(right\_answer }\SpecialCharTok{\textasciitilde{}} \FunctionTok{as.numeric}\NormalTok{(pas), }\AttributeTok{data =} \FunctionTok{filter}\NormalTok{(df\_exp1, subject }\SpecialCharTok{==} \DecValTok{7}\NormalTok{), }\AttributeTok{family =} \FunctionTok{binomial}\NormalTok{(}\AttributeTok{link =} \StringTok{"logit"}\NormalTok{))}
\NormalTok{x\_pas }\OtherTok{\textless{}{-}} \FunctionTok{seq}\NormalTok{(}\DecValTok{0}\NormalTok{, }\DecValTok{4}\NormalTok{, }\FloatTok{0.01}\NormalTok{)}
\NormalTok{y\_pas }\OtherTok{\textless{}{-}} \FunctionTok{predict}\NormalTok{(m7.}\FloatTok{1.1}\NormalTok{, }\FunctionTok{list}\NormalTok{(}\AttributeTok{pas =}\NormalTok{ x\_pas),}\AttributeTok{type=}\StringTok{"response"}\NormalTok{)}
\end{Highlighting}
\end{Shaded}

\begin{Shaded}
\begin{Highlighting}[]
\CommentTok{\#plots for subject 7}
\FunctionTok{plot}\NormalTok{(}\FunctionTok{as.numeric}\NormalTok{(df\_exp1}\SpecialCharTok{$}\NormalTok{pas), df\_exp1}\SpecialCharTok{$}\NormalTok{right\_answer, }\AttributeTok{pch =} \DecValTok{16}\NormalTok{, }\AttributeTok{xlab =} \StringTok{"Pas Rating"}\NormalTok{, }\AttributeTok{ylab =} \StringTok{"\% right answer = 1"}\NormalTok{, }
     \AttributeTok{xlim =} \FunctionTok{c}\NormalTok{(}\DecValTok{0}\NormalTok{,}\DecValTok{8}\NormalTok{)) }\SpecialCharTok{+} \FunctionTok{title}\NormalTok{(}\AttributeTok{main =} \StringTok{"subject 7 glm function with no interaction"}\NormalTok{, }\AttributeTok{sub =} \StringTok{"plot 2"}\NormalTok{)}
\end{Highlighting}
\end{Shaded}

\begin{verbatim}
## integer(0)
\end{verbatim}

\begin{Shaded}
\begin{Highlighting}[]
\FunctionTok{lines}\NormalTok{(x\_pas, y\_pas)}
\end{Highlighting}
\end{Shaded}

\includegraphics{practical_exercise_5_files/figure-latex/unnamed-chunk-39-1.pdf}

\begin{verbatim}
iii. Create a similar plot for the PAS ratings on a single plot (for subject 7), but this time based on the model from 6.1 `xlim=c(0, 8)`   
\end{verbatim}

\begin{Shaded}
\begin{Highlighting}[]
\NormalTok{df\_exp1\_sub7 }\OtherTok{\textless{}{-}}\NormalTok{ df\_exp1 }\SpecialCharTok{\%\textgreater{}\%} 
  \FunctionTok{filter}\NormalTok{(subject }\SpecialCharTok{==} \DecValTok{7}\NormalTok{)}

\NormalTok{lul\_model }\OtherTok{\textless{}{-}} \FunctionTok{glm}\NormalTok{(right\_answer }\SpecialCharTok{\textasciitilde{}}\NormalTok{ pas}\SpecialCharTok{*}\NormalTok{target.frames, }\AttributeTok{data =}\NormalTok{ df\_exp1\_sub7, }\AttributeTok{family =} \FunctionTok{binomial}\NormalTok{(}\AttributeTok{link =} \StringTok{"logit"}\NormalTok{))}
\end{Highlighting}
\end{Shaded}

\begin{Shaded}
\begin{Highlighting}[]
\NormalTok{df\_exp1\_sub7 }\SpecialCharTok{\%\textgreater{}\%} 
  \FunctionTok{ggplot}\NormalTok{(}\FunctionTok{aes}\NormalTok{(}\AttributeTok{x =}\NormalTok{ target.frames, }\AttributeTok{y =} \FunctionTok{fitted}\NormalTok{(lul\_model), }\AttributeTok{col =}\NormalTok{ pas)) }\SpecialCharTok{+} \FunctionTok{geom\_line}\NormalTok{() }\SpecialCharTok{+} \FunctionTok{ylim}\NormalTok{(}\DecValTok{0}\NormalTok{,}\DecValTok{1}\NormalTok{) }\SpecialCharTok{+} 
  \FunctionTok{labs}\NormalTok{(}\AttributeTok{title =} \StringTok{"Plot for subject 7, plot 3"}\NormalTok{, }\AttributeTok{y =} \StringTok{"\% of right\_answer = 1"}\NormalTok{)}
\end{Highlighting}
\end{Shaded}

\includegraphics{practical_exercise_5_files/figure-latex/unnamed-chunk-41-1.pdf}

\begin{Shaded}
\begin{Highlighting}[]
\NormalTok{newdat }\OtherTok{\textless{}{-}} \FunctionTok{data.frame}\NormalTok{(}\FunctionTok{cbind}\NormalTok{(}\StringTok{\textquotesingle{}target.frames\textquotesingle{}} \OtherTok{=} \FunctionTok{seq}\NormalTok{(}\DecValTok{0}\NormalTok{, }\DecValTok{8}\NormalTok{, }\AttributeTok{by =} \FloatTok{0.001}\NormalTok{), }\StringTok{\textquotesingle{}pas\textquotesingle{}} \OtherTok{=} \FunctionTok{rep}\NormalTok{(}\DecValTok{1}\SpecialCharTok{:}\DecValTok{4}\NormalTok{), }\StringTok{\textquotesingle{}subject\textquotesingle{}} \OtherTok{=} \FunctionTok{rep}\NormalTok{(}\StringTok{\textquotesingle{}7\textquotesingle{}}\NormalTok{)))}
\end{Highlighting}
\end{Shaded}

\begin{verbatim}
## Warning in cbind(target.frames = seq(0, 8, by = 0.001), pas = rep(1:4), : number
## of rows of result is not a multiple of vector length (arg 2)
\end{verbatim}

\begin{Shaded}
\begin{Highlighting}[]
\NormalTok{newdat}\SpecialCharTok{$}\NormalTok{subject }\OtherTok{\textless{}{-}} \FunctionTok{as.factor}\NormalTok{(newdat}\SpecialCharTok{$}\NormalTok{subject)}
\NormalTok{newdat}\SpecialCharTok{$}\NormalTok{pas }\OtherTok{\textless{}{-}} \FunctionTok{as.factor}\NormalTok{(newdat}\SpecialCharTok{$}\NormalTok{pas)}
\NormalTok{newdat}\SpecialCharTok{$}\NormalTok{target.frames }\OtherTok{\textless{}{-}} \FunctionTok{as.numeric}\NormalTok{(newdat}\SpecialCharTok{$}\NormalTok{target.frames)}

\NormalTok{newdat}\SpecialCharTok{$}\NormalTok{yhat }\OtherTok{\textless{}{-}} \FunctionTok{predict}\NormalTok{(m5.}\FloatTok{3.1}\NormalTok{, }\AttributeTok{newdata =}\NormalTok{ newdat, }\AttributeTok{type =} \StringTok{"response"}\NormalTok{)}
\end{Highlighting}
\end{Shaded}

\begin{Shaded}
\begin{Highlighting}[]
\FunctionTok{ggplot}\NormalTok{(newdat, }\FunctionTok{aes}\NormalTok{(}\AttributeTok{x =}\NormalTok{ target.frames, }\AttributeTok{y =}\NormalTok{ yhat, }\AttributeTok{color =}\NormalTok{ pas)) }\SpecialCharTok{+} \FunctionTok{geom\_line}\NormalTok{() }\SpecialCharTok{+} \FunctionTok{labs}\NormalTok{(}\AttributeTok{title =} \StringTok{"Plot for subject 7, plot 4"}\NormalTok{, }\AttributeTok{y =} \StringTok{"\% of right\_answer = 1"}\NormalTok{)}
\end{Highlighting}
\end{Shaded}

\includegraphics{practical_exercise_5_files/figure-latex/unnamed-chunk-43-1.pdf}

\begin{verbatim}
iv. Comment on the differences between the fits - mention some advantages and disadvantages of each way 
\end{verbatim}

\textbf{Comparing plot1 (sigmoid) and plot2 (subject 7):} \textbf{Plot
1} shows that as PAS approaches 4 the probability of guessing right
approaches 100\%. This is also illustrated in \textbf{plot 3} but also
adds the detail of how PAS4 interacts with target.frames. While higher
level of target.frames along with PAS4 rating will result in the best
probability of guessing right.

While \textbf{plot 1} gives us the best idea of how PAS rating effects
probability of guessing correct. \textbf{plot 3} is more informative
when it comes to illustrating the combination of target.frames and PAS.
But as we will show in the next exercise there is a way to combine the
information from both plots into 1 single plot.

\begin{enumerate}
\def\labelenumi{\arabic{enumi})}
\setcounter{enumi}{1}
\item
  Finally, estimate the parameters for all subjects and each of their
  four PAS ratings. Then plot the estimated function at the group-level
  by taking the mean for each of the four parameters, \emph{a},
  \emph{b}, \emph{c} and \emph{d} across subjects. A function should be
  estimated for each PAS-rating (it should look somewhat similar to Fig.
  3 from the article:
  \url{https://doi.org/10.1016/j.concog.2019.03.007})

  \begin{enumerate}
  \def\labelenumii{\roman{enumii}.}
  \tightlist
  \item
    compare with the figure you made in 5.3.ii and comment on the
    differences between the fits - mention some advantages and
    disadvantages of both.
  \end{enumerate}
\end{enumerate}

\begin{Shaded}
\begin{Highlighting}[]
\CommentTok{\#Create a function with a loop for calculating a,b,c and d for every subject within every level of PAS. }

\NormalTok{optim\_for\_indi }\OtherTok{\textless{}{-}} \ControlFlowTok{function}\NormalTok{()\{}
\NormalTok{  data\_frame\_parameters }\OtherTok{\textless{}{-}} \FunctionTok{data.frame}\NormalTok{(}\AttributeTok{subject =} \ConstantTok{NA}\NormalTok{, }\AttributeTok{pas =} \ConstantTok{NA}\NormalTok{ , }\AttributeTok{a =} \ConstantTok{NA}\NormalTok{, }\AttributeTok{b =} \ConstantTok{NA}\NormalTok{, }\AttributeTok{c =} \ConstantTok{NA}\NormalTok{, }\AttributeTok{d =} \ConstantTok{NA}\NormalTok{)}
  
  \ControlFlowTok{for}\NormalTok{ (i }\ControlFlowTok{in} \DecValTok{1}\SpecialCharTok{:}\DecValTok{4}\NormalTok{)\{}
\NormalTok{    df\_temp1 }\OtherTok{\textless{}{-}}\NormalTok{ df\_exp1 }\SpecialCharTok{\%\textgreater{}\%} 
      \FunctionTok{filter}\NormalTok{(pas }\SpecialCharTok{==}\NormalTok{ i)}
    
    \ControlFlowTok{for}\NormalTok{ (ii }\ControlFlowTok{in} \DecValTok{1}\SpecialCharTok{:}\FunctionTok{length}\NormalTok{(}\FunctionTok{unique}\NormalTok{(df\_exp1}\SpecialCharTok{$}\NormalTok{subject)))\{}
\NormalTok{      data\_temp }\OtherTok{\textless{}{-}}\NormalTok{ df\_temp1 }\SpecialCharTok{\%\textgreater{}\%} 
        \FunctionTok{filter}\NormalTok{(subject }\SpecialCharTok{==}\NormalTok{ ii) }\SpecialCharTok{\%\textgreater{}\%}
\NormalTok{        dplyr}\SpecialCharTok{::}\FunctionTok{select}\NormalTok{(target.frames, right\_answer, pas) }\SpecialCharTok{\%\textgreater{}\%} 
        \FunctionTok{rename}\NormalTok{(}\AttributeTok{x =}\NormalTok{ target.frames, }\AttributeTok{y =}\NormalTok{ right\_answer)}
        
\NormalTok{      op\_temp }\OtherTok{=} \FunctionTok{optim}\NormalTok{(}\AttributeTok{par =} \FunctionTok{c}\NormalTok{(}\FloatTok{0.5}\NormalTok{,}\FloatTok{0.5}\NormalTok{,}\DecValTok{1}\NormalTok{,}\DecValTok{1}\NormalTok{), }\AttributeTok{fn =}\NormalTok{ RSS, }\AttributeTok{data =}\NormalTok{ data\_temp, }\AttributeTok{method =} \StringTok{"L{-}BFGS{-}B"}\NormalTok{, }\AttributeTok{lower =} \FunctionTok{c}\NormalTok{(}\FloatTok{0.5}\NormalTok{,}\FloatTok{0.5}\NormalTok{,}\SpecialCharTok{{-}}\ConstantTok{Inf}\NormalTok{,}\SpecialCharTok{{-}}\ConstantTok{Inf}\NormalTok{),}
      \AttributeTok{upper =} \FunctionTok{c}\NormalTok{(}\DecValTok{1}\NormalTok{,}\DecValTok{1}\NormalTok{,}\ConstantTok{Inf}\NormalTok{,}\ConstantTok{Inf}\NormalTok{))}
\NormalTok{      data\_frame\_parameters }\OtherTok{\textless{}{-}} \FunctionTok{rbind}\NormalTok{(data\_frame\_parameters , }\FunctionTok{c}\NormalTok{(ii,i,op\_temp}\SpecialCharTok{$}\NormalTok{par))}
    
\NormalTok{    \}}
\NormalTok{  \}}
  \FunctionTok{return}\NormalTok{(data\_frame\_parameters)}
\NormalTok{\}}
\end{Highlighting}
\end{Shaded}

\begin{Shaded}
\begin{Highlighting}[]
\CommentTok{\#Get all parameters for each subject in each level of pas}
\NormalTok{df\_param }\OtherTok{\textless{}{-}} \FunctionTok{optim\_for\_indi}\NormalTok{() }\SpecialCharTok{\%\textgreater{}\%} 
  \FunctionTok{na.omit}\NormalTok{()}
\NormalTok{df\_param}
\end{Highlighting}
\end{Shaded}

\begin{verbatim}
##     subject pas         a         b         c             d
## 2         1   1 0.5000000 0.5000000 1.0000000  1.0000000000
## 3         2   1 0.6072665 0.5000000 0.9206736  0.7154031594
## 4         3   1 0.6248181 0.5374096 0.8602455  0.4300550152
## 5         4   1 0.7593121 0.5000000 0.9151872  0.0932777361
## 6         5   1 0.5000000 0.5000000 1.0000000  1.0000000000
## 7         6   1 0.6143188 0.5000000 1.0058600  0.9558662745
## 8         7   1 0.5000000 0.5000000 1.0000000  1.0000000000
## 9         8   1 0.6165075 0.5000000 0.9436198  0.8557934710
## 10        9   1 0.5000000 0.5391376 1.7117410  0.0203760864
## 11       10   1 0.5000000 0.5000000 1.0000000  1.0000000000
## 12       11   1 0.8903981 0.5060240 0.9271873  0.0992863965
## 13       12   1 0.5497749 0.5000000 0.9043642  0.5408935929
## 14       13   1 0.5833333 0.5000000 2.9701667  0.0735683037
## 15       14   1 0.5000000 0.5000000 1.0000000  1.0000000000
## 16       15   1 1.0000000 0.5000000 0.9581202  0.0776958221
## 17       16   1 0.5314009 0.5000000 2.8583552  0.0206781521
## 18       17   1 0.5000000 1.0000000 4.7067181  0.0351045146
## 19       18   1 0.5000000 0.5000000 1.0000000  1.0000000000
## 20       19   1 0.5000000 0.5000000 1.0000000  1.0000000000
## 21       20   1 0.5000000 0.5749949 2.8059813  0.0125588286
## 22       21   1 0.5000000 0.5000000 1.0000000  1.0000000000
## 23       22   1 0.6214298 0.5000000 1.9500852  0.0889697640
## 24       23   1 0.5000000 0.5985390 1.5230227  0.0361092523
## 25       24   1 0.5000000 0.5538462 3.6278255  0.0212250805
## 26       25   1 0.5000000 0.5000000 1.0000000  1.0000000000
## 27       26   1 0.5000000 0.6818167 2.0533035  0.0031743915
## 28       27   1 0.5542231 0.5000000 2.9327121  0.0041206099
## 29       28   1 0.5695365 1.0000000 3.0847012  0.0813136269
## 30       29   1 0.7873396 0.5000000 0.8659157  0.0919274040
## 31        1   2 0.5730489 0.9153286 3.5877061  0.0338588862
## 32        2   2 0.5000000 0.8888859 2.9648121  0.0691746862
## 33        3   2 0.5000000 0.5762683 1.9070036  0.0061812335
## 34        4   2 0.5000000 0.9875025 3.6355024  0.5948371098
## 35        5   2 0.5000000 1.0000000 2.9765903  0.0695711385
## 36        6   2 0.5000000 0.8935823 3.1013007  0.2984877652
## 37        7   2 0.5333448 0.6111099 2.0027946  0.0687175769
## 38        8   2 0.5000000 0.8771918 2.0215667  0.0435784343
## 39        9   2 0.5157711 0.8374833 2.4791204  0.3975403930
## 40       10   2 0.5000000 1.0000000 2.5493037  0.5108625890
## 41       11   2 0.5000000 0.9191919 2.2886839  0.0169055833
## 42       12   2 0.5887948 0.6962202 3.4166400  0.0258000759
## 43       13   2 0.5857144 1.0000000 3.0037711  0.0125240365
## 44       14   2 0.5833346 0.6904684 2.8112677  0.0119815065
## 45       15   2 0.5509160 0.9991819 4.0083535  0.0022637598
## 46       16   2 0.5000000 0.9429601 2.3411505  0.5384506750
## 47       17   2 0.5000000 0.5652205 1.1641578  0.0108894962
## 48       18   2 0.6714279 0.9772726 2.9737687  0.0693226975
## 49       19   2 0.5000000 0.9512212 4.0275957  0.0743356654
## 50       20   2 0.5000000 0.8888871 2.9555632  0.0755949310
## 51       21   2 0.5000000 0.9717277 2.3138548  0.3474994353
## 52       22   2 0.5000000 0.7646709 2.8742481  0.0398491958
## 53       23   2 0.5000000 1.0000000 1.8790847  0.7209948357
## 54       24   2 0.5104884 0.7300013 3.9527811  0.0746394535
## 55       25   2 0.5726492 1.0000000 3.0850966  0.0850103909
## 56       26   2 0.5000000 0.8943349 2.3150196  0.2176040986
## 57       27   2 0.5000000 0.9348424 2.6596019  0.2353014671
## 58       28   2 0.5000000 0.9746766 3.0911726  0.2560522320
## 59       29   2 0.6093749 0.9604134 3.6759258  0.1894237563
## 60        1   3 0.7819323 0.9755843 3.0052524  0.0007470902
## 61        2   3 0.5000000 0.9583337 2.3598420  0.0215939960
## 62        3   3 0.5000000 0.8684217 1.7035787  0.0858084103
## 63        4   3 0.5000000 0.9668874 3.4262634  0.0017956577
## 64        5   3 0.6363636 0.9090909 2.5123106  0.0029588883
## 65        6   3 0.5909041 1.0000000 2.9519210  0.0861951172
## 66        7   3 0.5000000 0.8947368 1.4342177  0.0231689981
## 67        8   3 0.5000000 0.9821427 2.4061812  0.0337914604
## 68        9   3 0.5403146 0.9313978 2.3805159  0.4296983390
## 69       10   3 0.5000000 0.9516040 2.2102736  0.3821491309
## 70       11   3 0.5000000 0.9922179 2.4333974  0.0153934467
## 71       12   3 0.9703795 0.9516391 0.9999966  0.9999466099
## 72       13   3 0.5000000 0.9973090 1.0181950  1.0560223608
## 73       14   3 0.5000000 0.9200017 3.0060639  0.0636149539
## 74       15   3 0.5423061 0.9472600 3.0152846  0.3230701375
## 75       16   3 1.0000000 0.9589040 4.0763834  0.0746820933
## 76       17   3 0.5000000 0.9523948 3.0310582  0.0576038444
## 77       18   3 0.5000000 0.9783654 1.2594758  0.0087766949
## 78       19   3 0.5000000 0.9605921 2.9544698  0.0866032228
## 79       20   3 0.5000000 0.9748576 2.0131601  0.0370179526
## 80       21   3 0.9804200 0.9934318 1.0000055  1.0000439608
## 81       22   3 0.5000000 0.9569231 1.9918145  0.0041758050
## 82       23   3 0.5000000 0.9776119 2.2664094  0.0005239727
## 83       24   3 0.5000000 0.8835090 3.5086789  0.3220243099
## 84       25   3 0.6666309 0.9512165 1.9558130  0.0824817618
## 85       26   3 0.5206045 0.9333405 2.8236588  0.1217143389
## 86       27   3 0.5000000 0.9727610 1.2453336  0.6980335530
## 87       28   3 0.5000000 0.9839587 1.7016998  0.5309014467
## 88       29   3 0.5000000 0.9886792 1.4696139  0.0011487401
## 89        1   4 1.0000000 0.9285775 5.0887524 -0.0056380228
## 90        2   4 0.7000000 0.9693396 2.8883605  0.0019708982
## 91        3   4 0.5000000 0.9978221 1.8170598  0.1055313760
## 92        4   4 0.7861323 1.0000000 0.9027309  0.5135729760
## 93        5   4 0.5000000 0.9975361 2.4314014  0.0249032159
## 94        6   4 0.6666791 0.9839973 2.7937806  0.0160633018
## 95        7   4 0.5000000 0.9900973 1.1927620  0.4163179741
## 96        8   4 0.5000000 0.9846507 1.3866717  0.0238588187
## 97        9   4 0.5000000 0.9302322 2.8886226  0.0702017095
## 98       10   4 1.0000000 1.0000000 1.0000000  1.0000000000
## 99       11   4 0.5000000 0.9998691 2.6934210  0.0003256725
## 100      12   4 1.0000000 1.0000000 0.9999921  0.9999710993
## 101      13   4 1.0000000 0.9934959 1.5819517  1.3508255145
## 102      14   4 0.5905771 0.9581773 3.4017789  0.3399522701
## 103      15   4 0.8139813 0.9976453 3.5713627  0.3930680869
## 104      16   4 1.0000000 0.9743435 1.3236845  1.2665296601
## 105      17   4 0.5000000 0.9924435 2.8950634  0.0816734101
## 106      18   4 0.5000000 1.0000000 2.6808485  0.0286670524
## 107      19   4 0.5000000 0.9851858 3.0813718  0.0768871749
## 108      20   4 0.6849777 1.0000000 0.9126481  0.5631541758
## 109      21   4 1.0000000 1.0000000 1.0000000  1.0000000000
## 110      22   4 0.5607111 0.9935075 3.0855950  0.0759710549
## 111      23   4 1.0000000 0.9969857 1.5023625  1.7249151316
## 112      24   4 0.5000000 0.5000000 1.0000000  1.0000000000
## 113      25   4 1.0000000 0.9919774 1.9280519  2.3566239423
## 114      26   4 1.0000000 0.8799920 3.1346974  0.0174575051
## 115      27   4 0.9333336 0.9722223 4.1078463  0.0638288255
## 116      28   4 0.6410124 0.9993099 1.0000009  1.0005706753
## 117      29   4 0.5000000 0.9922481 1.2165569  0.0222855873
\end{verbatim}

\begin{Shaded}
\begin{Highlighting}[]
\CommentTok{\#Compute the average value of a,b,c and d. }
\NormalTok{df\_param\_avg }\OtherTok{\textless{}{-}}\NormalTok{ df\_param }\SpecialCharTok{\%\textgreater{}\%} 
  \FunctionTok{group\_by}\NormalTok{(pas) }\SpecialCharTok{\%\textgreater{}\%}
  \FunctionTok{summarise}\NormalTok{(}\AttributeTok{a =} \FunctionTok{mean}\NormalTok{(a), }\AttributeTok{b =} \FunctionTok{mean}\NormalTok{(b), }\AttributeTok{c =} \FunctionTok{mean}\NormalTok{(c), }\AttributeTok{d =} \FunctionTok{mean}\NormalTok{(d))}

\NormalTok{df\_param\_avg}
\end{Highlighting}
\end{Shaded}

\begin{verbatim}
## # A tibble: 4 x 5
##     pas     a     b     c     d
##   <dbl> <dbl> <dbl> <dbl> <dbl>
## 1     1 0.580 0.551  1.64 0.457
## 2     2 0.527 0.878  2.83 0.176
## 3     3 0.577 0.956  2.28 0.226
## 4     4 0.720 0.966  2.19 0.501
\end{verbatim}

We currently have a problem of \emph{a} \textgreater{} \emph{b} in PAS1.
This will result in a declining sigmoid function. However, I can not
specify logical statement in lower\_boundaries ensuring \emph{b}
\textgreater{} \emph{a}.

\begin{Shaded}
\begin{Highlighting}[]
\CommentTok{\#x{-}values}
\NormalTok{x\_target.frames }\OtherTok{\textless{}{-}} \FunctionTok{seq}\NormalTok{(}\DecValTok{0}\NormalTok{,}\DecValTok{8}\NormalTok{,}\FloatTok{0.01}\NormalTok{)}


\CommentTok{\#setup data frame for x and y of different PAS score. }
\NormalTok{df\_x\_y }\OtherTok{\textless{}{-}} \FunctionTok{data.frame}\NormalTok{(x\_target.frames)}

\CommentTok{\#Function for estimating y\_hat given our a,b,c and d from our optim function. }
\NormalTok{y\_hat\_func }\OtherTok{\textless{}{-}} \ControlFlowTok{function}\NormalTok{()\{}
\ControlFlowTok{for}\NormalTok{ (i }\ControlFlowTok{in} \DecValTok{1}\SpecialCharTok{:}\DecValTok{4}\NormalTok{)\{}
\NormalTok{  y\_hat\_temp }\OtherTok{\textless{}{-}} \FunctionTok{sigmoid}\NormalTok{(df\_param\_avg}\SpecialCharTok{$}\NormalTok{a[i], df\_param\_avg}\SpecialCharTok{$}\NormalTok{b[i], df\_param\_avg}
                        \SpecialCharTok{$}\NormalTok{c[i], df\_param\_avg}\SpecialCharTok{$}\NormalTok{d[i], x\_target.frames)}
\NormalTok{  df\_x\_y[,i}\SpecialCharTok{+}\DecValTok{1}\NormalTok{] }\OtherTok{\textless{}{-}}\NormalTok{ y\_hat\_temp}
\NormalTok{\}}
  
\NormalTok{  df\_x\_y }\OtherTok{\textless{}{-}}\NormalTok{ df\_x\_y }\SpecialCharTok{\%\textgreater{}\%} \CommentTok{\#Rename column names into something meaningful. }
    \FunctionTok{rename}\NormalTok{(}\AttributeTok{x =}\NormalTok{ x\_target.frames, }\AttributeTok{y\_hat\_pas1 =}\NormalTok{ V2 , }\AttributeTok{y\_hat\_pas2 =}\NormalTok{ V3 , }\AttributeTok{y\_hat\_pas3 =}\NormalTok{ V4 , }\AttributeTok{y\_hat\_pas4 =}\NormalTok{ V5)}
  \FunctionTok{return}\NormalTok{(df\_x\_y)}
\NormalTok{\}}

\NormalTok{df\_final\_param }\OtherTok{\textless{}{-}} \FunctionTok{y\_hat\_func}\NormalTok{()}
\end{Highlighting}
\end{Shaded}

\begin{Shaded}
\begin{Highlighting}[]
\FunctionTok{head}\NormalTok{(df\_final\_param)}
\end{Highlighting}
\end{Shaded}

\begin{verbatim}
##      x y_hat_pas1 y_hat_pas2 y_hat_pas3 y_hat_pas4
## 1 0.00  0.5788822  0.5274092  0.5769072  0.7229803
## 2 0.01  0.5788658  0.5274092  0.5769079  0.7230414
## 3 0.02  0.5788491  0.5274092  0.5769086  0.7231037
## 4 0.03  0.5788320  0.5274092  0.5769094  0.7231672
## 5 0.04  0.5788146  0.5274092  0.5769102  0.7232319
## 6 0.05  0.5787968  0.5274092  0.5769110  0.7232980
\end{verbatim}

\begin{Shaded}
\begin{Highlighting}[]
\FunctionTok{tail}\NormalTok{(df\_final\_param)}
\end{Highlighting}
\end{Shaded}

\begin{verbatim}
##        x y_hat_pas1 y_hat_pas2 y_hat_pas3 y_hat_pas4
## 796 7.95  0.5514403  0.8775394  0.9556266  0.9658477
## 797 7.96  0.5514403  0.8775394  0.9556266  0.9658478
## 798 7.97  0.5514403  0.8775394  0.9556266  0.9658478
## 799 7.98  0.5514403  0.8775394  0.9556266  0.9658479
## 800 7.99  0.5514403  0.8775394  0.9556266  0.9658479
## 801 8.00  0.5514403  0.8775394  0.9556266  0.9658480
\end{verbatim}

As X -\textgreater{} 8 our y\_hat\_pas1 estimates based on the sigmoid
function defined in earlier chunks will decrease. This does not make
sense what so ever but is due to \emph{a} and \emph{b} from our optim
function.

\begin{Shaded}
\begin{Highlighting}[]
\NormalTok{df\_final\_param\_long }\OtherTok{\textless{}{-}}\NormalTok{ df\_final\_param }\SpecialCharTok{\%\textgreater{}\%} 
  \FunctionTok{pivot\_longer}\NormalTok{(}\AttributeTok{cols =} \FunctionTok{c}\NormalTok{(y\_hat\_pas1,y\_hat\_pas2,y\_hat\_pas3,y\_hat\_pas4) , }\AttributeTok{names\_to =} \StringTok{"pas\_name"}\NormalTok{, }\AttributeTok{values\_to =} \StringTok{"y\_hat\_merged"}\NormalTok{)}

\NormalTok{df\_final\_param\_long}
\end{Highlighting}
\end{Shaded}

\begin{verbatim}
## # A tibble: 3,204 x 3
##        x pas_name   y_hat_merged
##    <dbl> <chr>             <dbl>
##  1  0    y_hat_pas1        0.579
##  2  0    y_hat_pas2        0.527
##  3  0    y_hat_pas3        0.577
##  4  0    y_hat_pas4        0.723
##  5  0.01 y_hat_pas1        0.579
##  6  0.01 y_hat_pas2        0.527
##  7  0.01 y_hat_pas3        0.577
##  8  0.01 y_hat_pas4        0.723
##  9  0.02 y_hat_pas1        0.579
## 10  0.02 y_hat_pas2        0.527
## # ... with 3,194 more rows
\end{verbatim}

\begin{Shaded}
\begin{Highlighting}[]
\FunctionTok{ggplot}\NormalTok{(df\_final\_param\_long, }\FunctionTok{aes}\NormalTok{(x, y\_hat\_merged, }\AttributeTok{colour =}\NormalTok{ pas\_name)) }\SpecialCharTok{+}
  \FunctionTok{geom\_line}\NormalTok{() }\SpecialCharTok{+} \FunctionTok{ylim}\NormalTok{(}\DecValTok{0}\NormalTok{,}\DecValTok{1}\NormalTok{) }\SpecialCharTok{+} \FunctionTok{labs}\NormalTok{(}\AttributeTok{x =} \StringTok{"target.frames"}\NormalTok{, }\AttributeTok{title =} \StringTok{"averaged parameters across subject"}\NormalTok{)}
\end{Highlighting}
\end{Shaded}

\includegraphics{practical_exercise_5_files/figure-latex/unnamed-chunk-49-1.pdf}

\hypertarget{extra-stuff}{%
\subparagraph{Extra stuff}\label{extra-stuff}}

Checking the difference between taking the mean() of every parameter
c(a,b,c,d) and running the optim on data only divided by PAS ratings.

\begin{Shaded}
\begin{Highlighting}[]
\CommentTok{\#Make function}

\NormalTok{optim\_pas\_overall }\OtherTok{\textless{}{-}} \ControlFlowTok{function}\NormalTok{(i)\{}
  \CommentTok{\#prepare correct subset of  data frame.}
\NormalTok{  pas\_temp }\OtherTok{\textless{}{-}}\NormalTok{ df\_exp1 }\SpecialCharTok{\%\textgreater{}\%} 
    \FunctionTok{filter}\NormalTok{(pas }\SpecialCharTok{==}\NormalTok{ i) }\SpecialCharTok{\%\textgreater{}\%} 
\NormalTok{    dplyr}\SpecialCharTok{::}\FunctionTok{select}\NormalTok{(target.frames, right\_answer) }\SpecialCharTok{\%\textgreater{}\%}
    \FunctionTok{rename}\NormalTok{(}\AttributeTok{x =}\NormalTok{ target.frames, }\AttributeTok{y =}\NormalTok{ right\_answer)}
\NormalTok{  op\_temp\_pas }\OtherTok{=} \FunctionTok{optim}\NormalTok{(}\AttributeTok{par =} \FunctionTok{c}\NormalTok{(}\FloatTok{0.5}\NormalTok{,}\DecValTok{1}\NormalTok{,}\DecValTok{1}\NormalTok{,}\DecValTok{1}\NormalTok{), }\AttributeTok{fn =}\NormalTok{ RSS, }\AttributeTok{data =}\NormalTok{ pas\_temp, }\AttributeTok{method =} \StringTok{"L{-}BFGS{-}B"}\NormalTok{, }\AttributeTok{lower =}  \FunctionTok{c}\NormalTok{(}\FloatTok{0.5}\NormalTok{,}\FloatTok{0.5}\NormalTok{,}\SpecialCharTok{{-}}\ConstantTok{Inf}\NormalTok{,}\SpecialCharTok{{-}}\ConstantTok{Inf}\NormalTok{), }\AttributeTok{upper =} \FunctionTok{c}\NormalTok{(}\DecValTok{1}\NormalTok{,}\DecValTok{1}\NormalTok{,}\ConstantTok{Inf}\NormalTok{,}\ConstantTok{Inf}\NormalTok{)) }
  
  \CommentTok{\# x and y values}
\NormalTok{  x\_pas\_optim }\OtherTok{\textless{}{-}} \FunctionTok{seq}\NormalTok{(}\DecValTok{0}\NormalTok{, }\DecValTok{8}\NormalTok{, }\FloatTok{0.01}\NormalTok{)}
\NormalTok{  y\_pas\_optim }\OtherTok{\textless{}{-}} \FunctionTok{sigmoid}\NormalTok{(op\_temp\_pas}\SpecialCharTok{$}\NormalTok{par[}\DecValTok{1}\NormalTok{],op\_temp\_pas}\SpecialCharTok{$}\NormalTok{par[}\DecValTok{2}\NormalTok{],}
\NormalTok{                       op\_temp\_pas}\SpecialCharTok{$}\NormalTok{par[}\DecValTok{3}\NormalTok{],op\_temp\_pas}\SpecialCharTok{$}\NormalTok{par[}\DecValTok{4}\NormalTok{],x\_pas\_optim) }
  \FunctionTok{return}\NormalTok{(y\_pas\_optim)}
\NormalTok{\}}
\end{Highlighting}
\end{Shaded}

\begin{Shaded}
\begin{Highlighting}[]
\CommentTok{\#Generate y\_hats with function for all pas ratings. }
\NormalTok{data\_frame\_y\_hats }\OtherTok{\textless{}{-}} \FunctionTok{tibble}\NormalTok{(}\AttributeTok{x =}\NormalTok{ x\_target.frames, }\AttributeTok{y\_hat\_pas1 =} \FunctionTok{optim\_pas\_overall}\NormalTok{(}\DecValTok{1}\NormalTok{)}
\NormalTok{                            , }\AttributeTok{y\_hat\_pas2 =} \FunctionTok{optim\_pas\_overall}\NormalTok{(}\DecValTok{2}\NormalTok{)}
\NormalTok{                            , }\AttributeTok{y\_hat\_pas3 =} \FunctionTok{optim\_pas\_overall}\NormalTok{(}\DecValTok{3}\NormalTok{)}
\NormalTok{                            , }\AttributeTok{y\_hat\_pas4 =} \FunctionTok{optim\_pas\_overall}\NormalTok{(}\DecValTok{4}\NormalTok{))}
\end{Highlighting}
\end{Shaded}

\begin{Shaded}
\begin{Highlighting}[]
\CommentTok{\#Pivot longer to merge into 1 columns with an identity column.}
\NormalTok{data\_frame\_y\_hats\_long }\OtherTok{\textless{}{-}}\NormalTok{ data\_frame\_y\_hats }\SpecialCharTok{\%\textgreater{}\%} 
  \FunctionTok{pivot\_longer}\NormalTok{(}\AttributeTok{cols =} \FunctionTok{c}\NormalTok{(y\_hat\_pas1,y\_hat\_pas2,y\_hat\_pas3,y\_hat\_pas4) , }\AttributeTok{names\_to =} \StringTok{"pas\_name"}\NormalTok{, }\AttributeTok{values\_to =} \StringTok{"y\_hat\_merged"}\NormalTok{)}

\CommentTok{\#plot}
\FunctionTok{ggplot}\NormalTok{(data\_frame\_y\_hats\_long, }\FunctionTok{aes}\NormalTok{(x, y\_hat\_merged, }\AttributeTok{colour =}\NormalTok{ pas\_name)) }\SpecialCharTok{+}
  \FunctionTok{geom\_line}\NormalTok{() }\SpecialCharTok{+} \FunctionTok{ylim}\NormalTok{(}\DecValTok{0}\NormalTok{,}\DecValTok{1}\NormalTok{) }\SpecialCharTok{+} \FunctionTok{labs}\NormalTok{(}\AttributeTok{x =} \StringTok{"target.frames"}\NormalTok{, }\AttributeTok{title =} \StringTok{"Parameters estimated grouped by PAS"}\NormalTok{)}
\end{Highlighting}
\end{Shaded}

\includegraphics{practical_exercise_5_files/figure-latex/unnamed-chunk-52-1.pdf}
\textbf{Disadvantage} This plot doesn't clearly depict the interaction
effect between target.frames and PAS rating. \textbf{Advantage} This
plot/method alows us to set a minimum of our y-values. I've specified
lower boundary of a = 0.5 (not allowing below chance). As this was our
goal with using the optim function. Though there are several arguments
against setting a minimum y-value. (It's cheating. :P )

The Optim function tries to minimise our RSS given our other parameters
(a,b,c,d). So if we really were to compare the model from optim with min
= 0.5 and our interaction model from exercise 5 we should look at diff
in RSS. Specifying absurd high/low as \emph{a} \& \emph{b} = 1 would
give incredible accuracy by may have inflated RSS and poor prediction
accuracy of unseen data.

\end{document}
